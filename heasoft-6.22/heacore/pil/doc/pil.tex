%
%Please keep all special lines
%
\documentclass[]{isdc_docs}
\usepackage{html}
\usepackage{isdc_part2}
\usepackage{epsf}

\begin{document}

\def\eg {{\it e.g., }}
\def\ie {{\it i.e., }}


%
%starttitle
%
\long\def\doctitle {Parameter Interface Library\\ Users Manual}
\long\def\docrefer {ISDC/PIL}
\long\def\docissue {1.8.5}
\long\def\docchang {PIL 1.8.5}
\long\def\docidate {19 SEP 2002}
\long\def\docprepa {
		J. Borkowski \\
		}

\long\def\docagree {}
\long\def\doccheck {
		T. Lock \dotfill\\\ %by e-mail, on
		}
\long\def\docappro {
		R. Walter	\dotfill\  %on
		}
\long\def\dochisto {
                \texttt{31 AUG 1998} &1.0 & PIL Users Manual released\\
                \texttt{19 MAR 1999} &1.2 & PIL Users Manual released\\
                \texttt{25 JUN 1999} &1.3 & PIL Users Manual released\\
                \texttt{12 FEB 2001} &1.4 & PIL Users Manual released\\
                \texttt{10 MAR 2002} &1.8 & PIL Users Manual released\\
                \texttt{06 MAY 2002} &1.8.2 & PIL Users Manual released\\
                \texttt{19 SEP 2002} &1.8.5 & PIL Users Manual released\\
                \texttt{17 AUG 2004} &1.9.8 & PIL Users Manual released\\
                }

\long\def\docredoc{
		\esasoftstd
		\isdccts{2.1}
                \isdcscmp{-}
                \isdcsqap{-}
                \isdcsvvp{-}
		}

\long\def\docgloss {
	\refANALYSISSOFTWARE
	\refCOMPONENT
        \refDELIVERYMODULE 
	\refDEVELOPMENTTEAM
	\refDRIVER
	\refEXPORTEDHEADERFILE
	\refIDENTIFIERS
	\refINTERNALHEADERFILE
	\refPARAMETERFILES
        \refPIPELINE
        \refREFERENCEPLATFORM 
        \refSYSTEMRELEASE 
	\refSYSTEMSOFTWARE
}
%
\begISDCTITLE
\tableofcontents
\vspace{2cm}
\newpage
\endISDCTITLE
%
%starttext
%
%%%%%%%%%%%%%%%%%%%%%%%%%%%%%%%%%%%%%%%%%%%%%%%%%%%%%%%%%%%%%%%%%%%%%%%%%%%%%%
\begin{abstract}
{\large
The Parameter Interface Library (PIL) is an ANSI C/C++ and Fortran 90 callable 
library which manages access to IRAF/XPI compatible parameter files.
In general each program (executable) written for ISDC will have its
parameter file. PIL gives standard set of functions (API)
which can be called by applications wanting to access parameter files.
PIL functions allow for : reading/writing individual parameters from
parameter file, automatic selection of server mode, (re)opening/closing of
parameter file, querying information about parameter file.
C/C++ bindings have (almost) one to one corresponding Fortran bindings.
Minor differences are result of different calling conventions in C/C++ and
Fortran 90.

\vspace{0.5cm}
Version 1.6.x of PIL adds support for enumerated values, fixed and variable
length vectors of values, and expansion of environment variables specified in
parameters.

\vspace{0.5cm}
This document describes version 1.8.5 of PIL
}
\end{abstract}
\newpage

%
% include other document sections
%

\isdcpart{Getting started}
%
%%%%%%%%%%%%%%%%%%%%%%%%%%%%%%%%%%%%%%%%%%%%%%%%%%%%%%%%%%%%%%%%%%%%%%%%%%%%%%%
%%%%%%%%%%%%%%%%%%%%%%%%%%%%%%%%%%%%%%%%%%%%%%%%%%%%%%%%%%%%%%%%%%%%%%%%%%%%%%%

This chapter explains how to build and install the PIL library from its
source code distribution. It also provides some initial help on programming
with PIL by reviewing the sample programs that come with the PIL distribution.

\section{Building the PIL library}

\subsection{Necessary tools}

\subsubsection{Hardware}

Version 1.8.2 of PIL (May 2002) compiles on the following 
architectures :

\begin{verbatim}
  sparc-sun-solaris    (32 / 64 bit)
  intel-gnu-linux      (32 bit)
\end{verbatim}

The library may compile/run on other architectures (we have reports of
successful compilation on HP-UX, mips-sgi-irix and alpha-dec-osf machines), 
but PIL development team has access only to those aforementioned. Furthermore
ISDC officially supports only "reference platform", which is defined
as sparc-sun-solaris running Solaris 8 with Forte 6 compiler set.
Refer to :

\begin{verbatim}
   http://isdc.unige.ch/index.cgi?Software+swplatform
\end{verbatim}

for more information.

\subsubsection{Software}

Version 1.8.2 of PIL was tested with following compilers :

\begin{verbatim}
on Solaris 8 :

  Sunsoft cc C compiler version 5.0, 5.1, 6.x (aka Forte 6)
  Sunsoft CC C++ compiler version 5.0, 5.1, 6.x (aka Forte 6)
  Sunsoft f90 compiler version 5.0, 5.1, 6.x (aka Forte 6)

on Linux kernel version 2.2/2.4 :

  gcc/g++ C/C++ compiler version 2.7.*, 2.8.*, 2.95.2, 3.0.*
  Fujitsu Fortran 95 Express Version 1.0 (now Lahey ver 5+)
\end{verbatim}

PIL library may be compilable with other compilers, however 
the PIL development team have access to only those aforementioned.
Not all combinations of C/C++ and F90 compilers were tested.

Active development and debugging is done on Solaris with some
work also on Linux.

Besides compilers, one needs the following system utilities :

\begin{verbatim}
  gnu make (ver. 3.79.1) (Sun's /usr/ccs/bin/make does _NOT_ work)
  gzip
  tar (or gtar but renamed to tar)
\end{verbatim}


\subsection{Unpacking distribution file and setting up directories structure}

PIL library is usually delivered as a part of support-sw package. It is
however possible to compile it as a standalone package.

After downloading the distribution file, one has to move it to empty directory
in which she/he has read and write access rights. Then the following
should be done :

\begin{verbatim}
  gzip -d pil-1.8.2.tar.gz
  tar xvf pil-1.8.2.tar
\end{verbatim}

First line uncompresses the distribution file, the second unpacks files from
tar-file, usually creating several files and subdirectories. Main directory
should contain among the others Makefile.in, makeisdc1.in, configure.in, 
configure files and autoconf subdirectory.


\subsection{Autoconfiguring Makefiles}

Before attempting autoconfiguration the directory tree has to be set to the
clean state. This is done with the following command :

\begin{verbatim}
  make distclean
\end{verbatim}

This command deletes any existing object files, executables produced by
build process, config.cache, Makefiles produced by previous run of configure
script.

Before  running  configure one may wish to review (and probably manually
edit) files makeisdc1.in and Makefile.in. makeisdc1 and Makefile
should not be edited since their contents will be overwritten by subsequent
run of \${ISDC\_ENV}/bin/ac\_stuff/configure script.

To autoconfigure PIL library one has to run :

\begin{verbatim}
  $ISDC_ENV/bin/ac_stuff/configure
\end{verbatim}

\${ISDC\_ENV}/bin/ac\_stuff/configure script adapts Makefile to C
and F90 compilers tastes. If there is no F90 compiler available configure
script disables compilation of F90 source files. If configure script is 
unable to automatically find correct C and F90 compiler one can set CC, F90,
CFLAGS and F90FLAGS environment variables to force configure script to
choose specific compilers/options. For example configure script tends to 
favor gcc over other C compilers. So if you have gcc and other C
compiler it will always choose gcc. If you want to use C compiler other then
gcc please type : 

\begin{verbatim}
      setenv CC name_of_your_C_compiler (cc is quite common)
\end{verbatim}

before running ./configure 

\paragraph{Notes\\}
{\it
On some architectures NAG F90 compiler requires -Qpath option to be
predefined in F90FLAGS. configure script is not smart enough
to figure out where NAG F90's libraries/executables reside (usually
not in PATH). Besides, NAG F90 compiler is not supported.\\
If CFLAGS and F90FLAGS setup by ./configure script are incorrect one 
may fix them directly in Makefile/makeisdc1. For instance, some people
want to have libraries compiled with debug option turned on ("-g"), 
and configure script does not always put "-g" option in CFLAGS/F90FLAGS.
Please keep in mind that any subsequent run of configure script will
overwrite any changes.
}

\subsection{Compilation of library and demo programs}

To compile PIL library and build demo executables one has to type :

\begin{verbatim}
   make 
\end{verbatim}

This will create libpil.a file and executables :pset, plist, pilcdemo,
pilfdemo among others.


\subsection{Installation of library}

To install PIL package under \${ISDC\_ENV} directory tree one has to type :

\begin{verbatim}
   make global_install
\end{verbatim}

For more information about make files and installation procedure refer to 
makefiles-2.4.4 manual (found on ISDC web serwer) or README.make file.

\subsection{Sample programs}

A small number of sample programs is distributed together with PIL library
(and built when compiling library). They are :

\begin{itemize}
\item
pset - PIL's equivalent for IRAF/XPI pset program. Sets value of parameter
in arbitrary parameter file.
\item
pget - PIL's equivalent for IRAF/XPI pget program. Read value of parameter
in arbitrary parameter file and display it on screen.
\item
plist - PIL's equivalent for IRAF/XPI plist program. Dumps contents of
parameter file on screen.
\item
pil\_lint - program checks given parameter file and reports any errors
encountered.
\item
pil\_gen\_c\_code - given parameter file program dumps (to stdout) source
of C program which can read that parameter file.
\item
pilcdemo - sample C application (see source code for more info)
\item
pilfdemo - sample F90 application (see source code for more info)
\item
cvector - test program for vector support
\item
ISDCcopy - sample CTS compliant application (use make ISDCcopy to
build it - requires support-sw installed)
\end{itemize}

Most of those demo program are for testing purposes, and do not perform 
any useful calculations. Refer to chapters \ref{PILRefCcalling} and
\ref{PILRefF90calling} for information how to use PIL library.

\section{IRAF parameter files}\label{PILRefParFiles}

The purpose of PIL library is to enable ISDC applications access IRAF
compatible parameter files. IRAF parameter file is a text file.
Every line describes single parameter. Format is as follows : 

\begin{verbatim}
    name,type,mode,default,min,max,prompt 
\end{verbatim}

\begin{itemize}

\item
{\tt name} : name of parameter 

\item
{\tt type} : type of parameter. Allowable values: 

\begin{itemize}
\item
{\tt b} : means parameter of boolean type

\item
{\tt i} : means parameter of integer type

\item
{\tt r} : means parameter of real type

\item
{\tt s} : means parameter of string type

\item
{\tt f} : for parameter of filename type. {\tt f} may be followed
by any combination of r/w/e/n meaning test for read access, write
access, presence of file, absence of file. Thus fw means test whether
file given as a value of parameter is writable. 
\end{itemize}

\item
{\tt mode} : mode of parameter. Allowable value is any reasonable
(it is: a/h/q/hl/ql) combination of : 

\begin{itemize}
\item
{\tt a/auto} : effective mode equals to the {\it value} of parameter named
{\tt mode} in parameter file. If that parameter is not found in parameter
file or is found invalid then the effective mode is 'hidden'. 

\item
{\tt h/hidden} : No questions are asked, unless default value is invalid
for given data type. for example "qwerty" is specified as a value
for integer parameter. Note that is PILOverrideQueryMode function
was called with argument set to PIL\_QUERY\_OVERRIDE then no questions
are asked even for parameters with invalid values.

\item
{\tt l/learn} : If application changed parameter value, new value will be
written to parameter file when application terminates.
Actually this takes places when application calls either PILClose()
or PILFlushParameters() (or CommonExit() when writing ISDC's CTS
compliant software).
If this flag is not specified then any changes to the parameter 
(via PILGetXXX or PILPutXXX) are lost and are not written to disk.

\item
{\tt q/query} : Always ask for parameter. The format of the prompt is :
prompt field from parameter file (parameter name if prompt fielf
is empty) followed by allowable range (if any) in <>, followed
by default value (if any) in [], followed by colon. Pressing RETURN
key alone accepts default value. If newly entered value (or default
value in the case RETURN key alone is pressed) is unacceptable library
prompts user to reenter value. If the value is overridden by command
line argument, then the PIL library does not prompt for that parameter
(if value is valid and within boudaries if any) 
Note that is PILOverrideQueryMode function
was called with argument set to PIL\_QUERY\_OVERRIDE then no questions
are asked ever (even for parameters with invalid values).

\end{itemize}

\item
{\tt default} : default value for parameter [this field is optional]. This can be:
yes/no/y/n for booleans, integer/real literals (ie. 123, -34567 for integers, 1.23,
1234, -45.3e-5 for reals) for integers/reals, and string literals for
strings and filenames. String literals can be: abcdef, "abcdef", 'abcdef'. 

\item
{\tt min} : minimum value allowable for parameter [this field is optional].
Range checking is enabled only if both {\tt min} and {\tt max} fields
are nonempty. See also {\tt max}.
When {\tt max} field is empty then {\tt min} field can also contain pipe
symbol (vertical bar)
separated list of allowable values for given parameter. This works for
integer, real, string and filename types. For instance :

\begin{verbatim}
  OutFile,f,ql,"copy.o",/dev/null|copy.o|dest.o,,"Enter output file name"
\end{verbatim}

specifies that OutFile parameter can take only 3 distinct values, it is
/dev/null, copy.o or dest.o. Furthermore, for string (and \_ONLY\_ for string,
this does \_NOT\_ apply for filename parameter types) parameter types,
case does not matter, and string returned by PILGetString is converted
to uppercase. Note, that automatic conversion to uppercase is only done when
enum list is specified for given parameter.

\item
{\tt max} : maximum value allowable for parameter [this field is optional].
Works for integer, real string and filename data types. Both {\tt min} and 
{\tt max} must be specified for range checking to be active. Also {\tt min}
cannot be larger than {\tt max}. In case of any format error in {\tt min} or
{\tt max} PIL assumes that no range checking is requested. See also 
{\tt min} for a description of enumerated value list.

\item
{\tt prompt} : short description of parameter to be displayed whenever
library asks for value. If none is given library will display parameters name. 

\end{itemize}

As an example :

\begin{verbatim}
      pressure,r,ql,1013,,,"Enter atmospheric pressure in hPa" 
\end{verbatim}

describes parameter named pressure of type real, mode = query + learn, with
default value = 1013, no range checking and prompt

\begin{verbatim}
   "Enter atmospheric pressure in hPa" 
\end{verbatim}

Empty lines and lines beginning with '\#' are considered to be comment lines. 

\section{How parameter files are named and where are they looked for ?}\label{PILRefWhereFiles}

Typically for every executable there is a corresponding parameter file. The
name of parameter file is the name of executable file plus ".par" suffix.
Thus executable :

\begin{verbatim}
  isdc_copy
\end{verbatim}

will have parameter file named :

\begin{verbatim}
  isdc_copy.par
\end{verbatim}

Parameter files are searched for in several locations: 

\begin{itemize}
\item
PFILES enviromnent variable 

This variable is used to specify where the parameters are looked for. The
variable uses a ";" delimiter to
separate 2 types of parameter directories: 

\begin{verbatim}
<path1>;<path2>
\end{verbatim}

where path 1 is one or more user/writeable parameter directories, and
path 2 is one or more system/read-only parameter directories. When
path 1 equals path 2 one can omit path 2 and semicolon. The PIL library
allows multiple ":"-delimited directories in both portions of the
PFILES variable. The default values from path 2 are used the first
time task is run, or whenever the default values have been
updated more recently than the user's copy of the parameters. The user's
copy is created when a task terminates, and retains any learned
changes to the parameters. 

\item
Current working directory.

It is equivalent of PFILES set to ".". 
\end{itemize}

\paragraph{Notes\\}
{\it
contrary to FTOOLS/SAOrd/XPI PFILES environment variable can be left
undefined. In this can case PIL
library will look for parameter files only in current directory. FTOOLS
executables return error in this situation. 
}


\section{PIL extensions to IRAF parameter files format}\label{PILRefExtensions}

This section lists PIL extensions to IRAF parameter files.

\subsection{Parameter file line length}\label{PILRefLineLength}

PIL limits parameter file length to 2000 characters. This is much larger
that IRAF parameter file line length limit (80 or 255 depending on
application/library). Thus parameter files created by PIL may not
necessarily be readable by IRAF.

\subsection{Expansion of environment variables}\label{PILRefEnvExpansion}

If value of given parameter contains the following string :

\begin{verbatim}
      ${ANY_VAR_NAME}
\end{verbatim}

PIL expands it to the value of ANY\_VAR\_NAME environment variable. Any number
of environment variables can be specified. Also the same variable can be
specified many times. As an example, for a user joe with home directory
in /home/joe :

\begin{verbatim}
      ${HOME}/pfiles/${HOME}
\end{verbatim}

is expanded to

\begin{verbatim}
      /home/joe/pfiles/home/joe
\end{verbatim}

\subsection{Vector support}\label{PILRefVector}

PIL internally supports parameters with values with specify vectors with
either fixed or variable number of elements. This works for parameter
types: integer, real and real4. The parameter itself has
to be of type string (type stored in parameter file). Thus it can be read 
as either string (by calling PILGetString) or vector of values
(by calling PILGetxxxVector or PILGetxxxVarVector). See description of
PILGetxxxVector or PILGetxxxVarVector functions for more details.


\section{How parameters are evaluated}\label{PILRefEvaluate}

In general parameter's values are read from the parameter file. They can
however be overridden if the command line arguments are entered. 
Let's assume that we have written simple application named ISDCCopy to 
copy a file. It accepts 2 parameters, namely InFile and OutFile. 
The corresponding parameter file could be : 
  
\begin{verbatim}
InFile,s,ql,sample1.fits,,,"Enter input file name" 
OutFile,s,ql,/dev/null,,,"Enter output file name" 
\end{verbatim}

If we run that application without any arguments 

\begin{verbatim}
      ISDCCopy
\end{verbatim}

it will prompt us for these two parameters. Pressing 2 times {\bf RETURN} key
will accept default values and in this case application will attempt to copy
sample1.fits file onto /dev/null device (not a very useful function).
If we type : 

\begin{verbatim}
      ISDCCopy sample34.fits
\end{verbatim}

or 

\begin{verbatim}
      ISDCCopy InFile="sample34.fits"
\end{verbatim}

it will ask only for 2nd second parameter. The value of first parameter will
be set to sample34.fits. If we
type: 

\begin{verbatim}
      ISDCCopy sample45.fits mycopy.fits
\end{verbatim}

or 

\begin{verbatim}
      ISDCCopy InFile="sample45.fits" OutFile="mycopy.fits"
\end{verbatim}

or 

\begin{verbatim}
      ISDCCopy OutFile="mycopy.fits" InFile="sample45.fits"
\end{verbatim}

it will not ask for any parameters. If we type : 

\begin{verbatim}
      ISDCCopy OutFile="mycopy.fits"
\end{verbatim}

it will ask only for the first parameter. 

\paragraph{Notes\\}
{\it
If a call to PILOverrideQueryMode has been made and PIL\_QUERY\_OVERRIDE
mode is in effect PIL library does not prompt user when invalid or out
of range argument is encountered. Instead it returns with an error.

Bogus parameter names passed in command line (i.e. those without maching
counterpart in parameter file) may result in application returning 
with an error PIL\_BOGUS\_CMDLINE. This can happen if application calls
PILVerifyCmdline(). PILInit() itself does not call PILVerifyCmdline().
}

\subsection{Positional parameters}\label{PILRefPositionalPar}

When specifying new value for a parameter on the command line, the
simplest method is to type its value only :

\begin{verbatim}
      exename par1value par2value par2value [ ... ]
\end{verbatim}

More specifically, PIL treats n-th command line argument as new value
for n-th parameter in the parameter file (hence the name *positional*). 
That is, assuming there are no empty/comment lines, the parameter in 
the n-th line in the parameter file.

\subsection{Named parameters}\label{PILRefNamedPar}

A more flexible method to specify new value of the parameter is to use
the syntax :

\begin{verbatim}
      name=value
\end{verbatim}

Using this syntax, one can specify new value for the parameters in any
order. Thus :

\begin{verbatim}
      ISDCCopy OutFile="mycopy.fits" InFile="sample45.fits"
\end{verbatim}

and

\begin{verbatim}
      ISDCCopy InFile="sample45.fits" OutFile="mycopy.fits"
\end{verbatim}

are equivalent. Furthermore, since spaces are allowed around '=' separator,
than all the following formats are equivalent :

\begin{verbatim}
      ISDCCopy OutFile=mycopy.fits
      ISDCCopy OutFile =mycopy.fits
      ISDCCopy OutFile= mycopy.fits
      ISDCCopy OutFile = mycopy.fits
\end{verbatim}

{\it note: any positional parameters must precede parameters given 
with name=value syntax.
}


\isdcpart{PIL C/C++ Language Interface}
\section{PIL C/C++ Language API}\label{PILRefClanguage}
%
%%%%%%%%%%%%%%%%%%%%%%%%%%%%%%%%%%%%%%%%%%%%%%%%%%%%%%%%%%%%%%%%%%%%%%%%%%%%%%%
%%%%%%%%%%%%%%%%%%%%%%%%%%%%%%%%%%%%%%%%%%%%%%%%%%%%%%%%%%%%%%%%%%%%%%%%%%%%%%%

\subsection{Introduction}\label{PILRefCintro}

This  section  describes  the  C language implementation of the
Parameter Interface Library Application Programming Interfaces (PIL APIs)
It also gives C/C++ languages specific
information necessary to use those APIs. Users interested only
in using PIL from Fortran 90 applications should read section
\ref{PILRefF90language}. 

PIL library is standalone and can be compiled independently of other ISDC
libraries.

Unless stated otherwise all PIL API functions return status code of type
int. This is either ISDC\_OK which equals PIL\_OK which equals 0, which
means everything went perfectly or negative value (error code) meaning some
error occured. List of error codes can be found in pil\_error.h file.
Functions given below are the "official" ones. Internally PIL library
calls many more functions. 	

\subsection{PIL C/C++ include files}\label{PILRefCincludes}

Applications calling PIL services from C/C++ source code should include
{\tt pil.h}. Alternatively they can include {\tt isdc.h} which includes
{\tt pil.h} file. {\tt pil.h} file in turn internally includes all other
PIL relevant include files.

{\tt pil.h} file can be called from either C or C++ source code. It contains
prototypes of all C/C++ API functions, definitions of constants,
declarations of global variables and definitions of data structures.


\subsection{C/C++ API functions}\label{PILRefCfunctions}

%%%%%%%%%%%          PILinit            %%%%%%%%%%%%%%%%%

\subsubsection{PILinit}

\begin{verbatim}
int PILInit(int argc, char **argv); 
\end{verbatim}

\paragraph{Description\\}
This function initializes PIL library. This function has to be called before
any other PIL function (there are some
exceptions to this rule).  It does the following : \\
Based on PILModuleName (or argv[0] if PILModuleName is empty) calculates
name of parameter file. Usually name
of parameter file equals argv[0] + ".par" suffix but this can be overriden
by calling PILSetModuleName and/or
PILSetModuleVersion before calling PILInit. After successful termination
parameter file is opened and read in, and
global variable PILRunMode is set to one of the following values : 

\begin{verbatim}
      ISDC_SINGLE_MODE 
      ISDC_SERVER_MODE 
\end{verbatim}

ISDC\_SERVER\_MODE is set whenever there is parameter "ServerMode" and its
value is "yes" or "y". In any other case PILRunMode is set
to ISDC\_SINGLE\_MODE. 

\paragraph{Return Value\\}
If success returns ISDC\_OK. Error code otherwise. See appendix \ref{PILRefErrorCodes}
for a list of error codes and their explanation.

\paragraph{Parameters}
\begin{itemize}
\item
{\tt int argc [In] } \\
number of command line arguments
\item
{\tt char **argv [In] } \\
array of pointers to command line arguments. argv[0] is typically name of
the executable.
\end{itemize}

\paragraph{Notes\\}
{\it
Only one parameter file can be open at a time. Parameter file remains open
until PILClose is called. When
writing applications for ISDC one should not use PILInit directly. Instead,
one should call CommonInit function
(from Common library) which calls PILInit. 
}


%%%%%%%%%%%       PILClose       %%%%%%%%%%%%%%%%%

\subsubsection{PILClose}

\begin{verbatim}
int PILClose(int status); 
\end{verbatim}

\paragraph{Description\\}
This function has to be called before application terminates. It closes open
files, and writes all learned parameters
to the disk files (only when status == ISDC\_OK). Once this function is
called one cannot call any other PIL
functions. One can however call PILInit to reinitialize PIL library. \\
Function also clears PILRunMode, PILModuleName and PILModuleVersion global
variables. 

\paragraph{Return Value\\}
If success returns ISDC\_OK. Error code otherwise. See appendix \ref{PILRefErrorCodes}
for a list of error codes and their explanation.

\paragraph{Parameters}
\begin{itemize}
\item
{\tt int status [In] } \\
PIL\_OK/ISDC\_OK/0 means normal return, any other value means abnormal
termination. In this case changes to parameters made during runtime 
are NOT written to parameter files. 
\end{itemize}

\paragraph{Notes\\}
{\it
This function does not terminate process. It simply shuts down PIL library.
When writing applications for ISDC one should not use PILClose directly. 
Instead, one should call CommonExit function (from Common library) 
which calls PILClose.
}


%%%%%%%%%%%       PILReloadParameters       %%%%%%%%%%%%%%%%%

\subsubsection{PILReloadParameters}

\begin{verbatim}
int PILReloadParameters(void);
\end{verbatim}

\paragraph{Description\\}
This function reloads parameters from parameter file. It is called
internally by PILInit. It
should be called explicitly by applications running in ISDC\_SERVER\_MODE to
rescan
parameter file and reload parameters from it. Current parameter list in
memory (including
any modifications) is deleted. PIL library locks whole file for exclusive
access when reading
from parameter file. 

\paragraph{Return Value\\}
If success returns ISDC\_OK. Error code otherwise. See appendix \ref{PILRefErrorCodes}
for a list of error codes and their explanation.

\paragraph{Notes\\}
{\it
parameter file remains open until PILClose is called. Function internally
DOES NOT close/reopen parameter file. Application should call
PILFlushParameters before calling this function, otherwise all changes
made to parameters so far are lost.
}


%%%%%%%%%%%       PILFlushParameters       %%%%%%%%%%%%%%%%%

\subsubsection{PILFlushParameters}

\begin{verbatim}
int PILFlushParameters(void);
\end{verbatim}

\paragraph{Description\\}
This function flushes changes made to parameter list (in memory) to disk.
Current contents
of parameter file is overwritten. PIL library locks whole file for exclusive
access when writing to parameter file.

\paragraph{Return Value\\}
If success returns ISDC\_OK. Error code otherwise. See appendix \ref{PILRefErrorCodes}
for a list of error codes and their explanation.

\paragraph{Notes\\}
{\it
parameter file remains open until PILClose is called.
}


%%%%%%%%%%%       PILSetModuleName       %%%%%%%%%%%%%%%%%

\subsubsection{PILSetModuleName}

\begin{verbatim}
int PILSetModuleName(char *name);
\end{verbatim}

\paragraph{Description\\}
Sets name of the module which uses PIL services. Result is stored in global
variable
PILModuleName. Usually name of parameter file equals argv[0] + ".par" suffix
but this can
be overriden by calling PILSetModuleName and/or PILSetModuleVersion before
calling
PILInit. 

\paragraph{Return Value\\}
If success returns ISDC\_OK. Error code otherwise. See appendix \ref{PILRefErrorCodes}
for a list of error codes and their explanation.

\paragraph{Parameters}
\begin{itemize}
\item
{\tt char *name [In] } \\
new module name
\end{itemize}


%%%%%%%%%%%       PILSetModuleVersion       %%%%%%%%%%%%%%%%%

\subsubsection{PILSetModuleVersion}

\begin{verbatim}
int PILSetModuleVersion(char *version);
\end{verbatim}

\paragraph{Description\\}
Sets version of the module which uses PIL services. Result is stored in
global variable
PILModuleVersion.

\paragraph{Return Value\\}
If success returns ISDC\_OK. Error code otherwise. See appendix \ref{PILRefErrorCodes}
for a list of error codes and their explanation.

\paragraph{Parameters}
\begin{itemize}
\item
{\tt char *version [In] } \\
new module version. If NULL pointer is passed version is set to "version
unspecified" string
\end{itemize}


%%%%%%%%%%%       PILGetParFilename       %%%%%%%%%%%%%%%%%

\subsubsection{PILGetParFileName}

\begin{verbatim}
int PILGetParFilename(char **fname);
\end{verbatim}

\paragraph{Description\\}
This function retrieves full path of used parameter file. Absolute path is
returned only when
PFILES environment variable contains absolute paths. If parameter file is
taken from
current dir then only filename is returned. Pointer returned points to a
statically allocated
buffer, applications should copy data from it using strcpy. 

\paragraph{Return Value\\}
If success returns ISDC\_OK. Error code otherwise. See appendix \ref{PILRefErrorCodes}
for a list of error codes and their explanation.

\paragraph{Parameters}
\begin{itemize}
\item
{\tt char **fname [Out] } \\
pointer to name/path of the parameter file 
\end{itemize}


%%%%%%%%%%%       PILOverrideQueryMode       %%%%%%%%%%%%%%%%%

\subsubsection{PILOverrideQueryMode}

\begin{verbatim}
int PILOverrideQueryMode(int newmode);
\end{verbatim}

\paragraph{Description\\}
This functions globally overrides query mode. When newmode passed is
PIL\_QUERY\_OVERRIDE, prompting for new values of parameters is completely
disabled. If
value is bad or out of range PILGetXXX return immediately with error without
asking user.
No i/o in stdin/stdout is done by PIL in this mode. When newmode is
PIL\_QUERY\_DEFAULT
PIL reverts to default query mode. 

\paragraph{Return Value\\}
If success returns ISDC\_OK. Error code otherwise. See appendix \ref{PILRefErrorCodes}
for a list of error codes and their explanation.

\paragraph{Parameters}
\begin{itemize}
\item
{\tt int newmode [In] } \\
new value for query override mode
\end{itemize}


%%%%%%%%%%%       PILGetBool       %%%%%%%%%%%%%%%%%

\subsubsection{PILGetBool}

\begin{verbatim}
int PILGetBool(char *name, int *result); 
\end{verbatim}

\paragraph{Description\\}
This function reads the value of specified parameter. The parameter has to
be of type
boolean. If this is not the case error code is returned. 

\paragraph{Return Value\\}
If success returns ISDC\_OK. Error code otherwise. See appendix \ref{PILRefErrorCodes}
for a list of error codes and their explanation.

\paragraph{Parameters}
\begin{itemize}
\item
{\tt char *name [In] } \\
name of the parameter 
\item
{\tt int *result [Out] } \\
pointer to integer variable which will store result. Boolean value of
FALSE is returned as a 0. Any other value means TRUE. 
\end{itemize}


%%%%%%%%%%%       PILGetInt       %%%%%%%%%%%%%%%%%

\subsubsection{PILGetInt}

\begin{verbatim}
int PILGetInt(char *name, int *result); 
\end{verbatim}

\paragraph{Description\\}
This function reads the value of specified parameter. The parameter has to
be of type
integer. If this is not the case error code is returned. 


\paragraph{Return Value\\}
If success returns ISDC\_OK. Error code otherwise. See appendix \ref{PILRefErrorCodes}
for a list of error codes and their explanation.

\paragraph{Parameters}
\begin{itemize}
\item
{\tt char *name [In]} \\
name of the parameter 
\item
{\tt int *result [Out]} \\
pointer to integer variable which will store result.

\end{itemize}


%%%%%%%%%%%       PILGetReal       %%%%%%%%%%%%%%%%%

\subsubsection{PILGetReal}

\begin{verbatim}
int PILGetReal(char *name, double *result); 
\end{verbatim}

\paragraph{Description\\}
This function reads the value of specified parameter. The parameter has to
be of type real. If this is not the case
error code is returned.

\paragraph{Return Value\\}
If success returns ISDC\_OK. Error code otherwise. See appendix \ref{PILRefErrorCodes}
for a list of error codes and their explanation.

\paragraph{Parameters}
\begin{itemize}
\item
{\tt char *name [In]} \\
name of the parameter
\item
{\tt double *result [Out]} \\
pointer to double variable which will store result. 
\end{itemize}


%%%%%%%%%%%       PILGetReal4       %%%%%%%%%%%%%%%%%

\subsubsection{PILGetReal4}

\begin{verbatim}
int PILGetReal4(char *name, float *result);
\end{verbatim}

\paragraph{Description\\}
This function reads the value of specified parameter. The parameter has to
be of type real. If this is not the case
error code is returned. 

\paragraph{Return Value\\}
If success returns ISDC\_OK. Error code otherwise. See appendix \ref{PILRefErrorCodes}
for a list of error codes and their explanation.

\paragraph{Parameters}
\begin{itemize}
\item
{\tt char *name [In] } \\
\item
name of the parameter 
{\tt float *result [Out] } \\
\end{itemize}
pointer to float variable which will store result.
\paragraph{Notes\\}
{\it
PIL library internally performs all computations using double data type.
This function merely calls PILGetReal() function then converts double to
float.
}


%%%%%%%%%%%       PILGetString       %%%%%%%%%%%%%%%%%

\subsubsection{PILGetString}

\begin{verbatim}
int PILGetString(char *name, char *result);
\end{verbatim}

\paragraph{Description\\}
This function reads the value of specified parameter. The parameter has to
be of type string. If this is not the case error code is returned. 
It is possible to enter empty string (without accepting default value).
By default entering string "" (two doublequotes) sets value of given string
parameter to an empty string. If PIL\_EMPTY\_STRING environment variable
is defined then its value is taken as an empty string equivalent.

\paragraph{Return Value\\}
If success returns ISDC\_OK. Error code otherwise. See appendix \ref{PILRefErrorCodes}
for a list of error codes and their explanation.

\paragraph{Parameters}
\begin{itemize}
\item
{\tt char *name [In] } \\
name of the parameter
\item
{\tt char *result [Out] } \\
pointer to character array which will store result. The character array
should have at least PIL\_LINESIZE characters to assure enough storage for the longest
possible string. 
\end{itemize}


\paragraph{Notes\\}
{\it
When enum list is specified for given parameter and is in effect (see
section \ref{PILRefParFiles}), then PILGetString converts value
entered to uppercase before returning. Also when comparing default/entered
value PILGetString ignores case. This (i.e. conversion to uppercase
and case-insensitive comparison) applies \_ONLY\_ to string parameters
and \_ONLY\_ to those for which pipe (vertical bar) separated enum list is 
specified (in {\tt min} field).
}


%%%%%%%%%%%       PILGetFname       %%%%%%%%%%%%%%%%%

\subsubsection{PILGetFname}

\begin{verbatim}
int PILGetFname(char *name, char *result); 
\end{verbatim}

\paragraph{Description\\}
This function reads the value of specified parameter. The parameter has to
be of type filename. If this is not the case error code is returned. 
It is possible to enter empty string (without accepting default value).
By default entering string "" (two doublequotes) sets value of given string
parameter to an empty string. If PIL\_EMPTY\_STRING environment variable
is defined then its value is taken as an empty string equivalent.

\paragraph{Return Value\\}
If success returns ISDC\_OK. Error code otherwise. See appendix \ref{PILRefErrorCodes}
for a list of error codes and their explanation.

\paragraph{Parameters}
\begin{itemize}
\item
{\tt char *name [In] } \\
name of the parameter
\item
{\tt char *result [Out] } \\
pointer to character array which will store result. The character array
should have at least PIL\_LINESIZE characters to assure enough storage for
the longest possible string.
\end{itemize}

\paragraph{Notes\\}
{\it
leading and trailing spaces are trimmed.
If the type of the parameter specifies it, access mode checks are done 
on file. So if access mode specifies write mode, and the file is read-only 
then PILGetFname will not accept that filename
and will prompt user to enter name of the file which is writable. 
Before applying any checks the value of the parameter (which may be URL, 
for instance http://, file://etc/passwd) is converted to the filename
(if it is possible). Details are given in paragraph \ref{PILSetRootNameFunction}.
}


%%%%%%%%%%%       PILGetDOL       %%%%%%%%%%%%%%%%%

\subsubsection{PILGetDOL}

\begin{verbatim}
int PILGetDOL(char *name, char *result); 
\end{verbatim}

\paragraph{Description\\}
This function reads the value of specified parameter. The parameter has to
be of type string. If this is not the case error code is returned. 
It is possible to enter empty string (without accepting default value).
By default entering string "" (two doublequotes) sets the value of given string
parameter to an empty string. If PIL\_EMPTY\_STRING environment variable
is defined then its value is taken as an empty string equivalent.

\paragraph{Return Value\\}
If success returns ISDC\_OK. Error code otherwise. See appendix \ref{PILRefErrorCodes}
for a list of error codes and their explanation.

\paragraph{Parameters}
\begin{itemize}
\item
{\tt char *name [In] } \\
name of the parameter
\item
{\tt char *result [Out] } \\
pointer to character array which will store result. The character array
should have at least PIL\_LINESIZE characters to assure enough storage for
the longest possible string.
\end{itemize}

\paragraph{Notes\\}
{\it
leading and trailing spaces are trimmed.
}


%%%%%%%%%%%       PILGetAsString       %%%%%%%%%%%%%%%%%

\subsubsection{PILGetAsString}

\begin{verbatim}
int PILGetAsString(const char *name, char *result);
\end{verbatim}

\paragraph{Description\\}
This function reads the value of specified parameter, regardless of its type,
and returns its value as a string.

\paragraph{Return Value\\}
If successful returns ISDC\_OK; otherwise returns an error code. See appendix \ref{PILRefErrorCodes}
for a list of error codes and their explanations.

\paragraph{Parameters}
\begin{itemize}
\item
{\tt char *name [In] } \\
name of the parameter
\item
{\tt char *result [Out] } \\
pointer to character array which will store result. The character array
should have at least PIL\_LINESIZE characters to assure enough storage for the longest
possible string. 
\end{itemize}


\paragraph{Notes\\}
{\it
The value of the parameter is obtained by calling the correct PIL code
for the actual parameter type. That is, if the parameter type is bool, the
code in PILGetBool will be called, if a string, PILGetString will be
called etc. Therefore, please consult the detailed description of the
relevant function for specific behavior for each parameter type.
}


%%%%%%%%%%%       PILGetIntVector       %%%%%%%%%%%%%%%%%

\subsubsection{PILGetIntVector}

\begin{verbatim}
int PILGetIntVector(char *name, int nelem, int *result); 
\end{verbatim}

\paragraph{Description\\}
This function reads the value of specified parameter. The parameter has
to be of type string (type stored in parameter file). If this is not the
case error code is returned. Ascii string (value of parameter) has to have
exactly NELEM integer numbers separated with spaces. If there are more
or less then NELEM then error code is returned.


\paragraph{Return Value\\}
If success returns ISDC\_OK. Error code otherwise. See appendix \ref{PILRefErrorCodes}
for a list of error codes and their explanation.

\paragraph{Parameters}
\begin{itemize}
\item
{\tt char *name [In]} \\
name of the parameter 
\item
{\tt int nelem [In]} \\
number of integers to return
\item
{\tt int *result [Out]} \\
pointer to vector of integers to store result

\end{itemize}


%%%%%%%%%%%       PILGetRealVector       %%%%%%%%%%%%%%%%%

\subsubsection{PILGetRealVector}

\begin{verbatim}
int PILGetRealVector(char *name, int nelem, double *result); 
\end{verbatim}

\paragraph{Description\\}
This function reads the value of specified parameter. The parameter has
to be of type string (type stored in parameter file). If this is not the
case error code is returned. Ascii string (value of parameter) has to have
exactly NELEM real numbers separated with spaces. If there are more
or less then NELEM then error code is returned.

\paragraph{Return Value\\}
If success returns ISDC\_OK. Error code otherwise. See appendix \ref{PILRefErrorCodes}
for a list of error codes and their explanation.

\paragraph{Parameters}
\begin{itemize}
\item
{\tt char *name [In]} \\
name of the parameter
\item
{\tt int nelem [In]} \\
number of doubles to return
\item
{\tt double *result [Out]} \\
pointer to vector of doubles to store result
\end{itemize}


%%%%%%%%%%%       PILGetReal4Vector       %%%%%%%%%%%%%%%%%

\subsubsection{PILGetReal4Vector}

\begin{verbatim}
int PILGetReal4Vector(char *name, int nelem, float *result);
\end{verbatim}

\paragraph{Description\\}
This function reads the value of specified parameter. The parameter has
to be of type string (type stored in parameter file). If this is not the
case error code is returned. Ascii string (value of parameter) has to have
exactly NELEM real numbers separated with spaces. If there are more
or less then NELEM then error code is returned.

\paragraph{Return Value\\}
If success returns ISDC\_OK. Error code otherwise. See appendix \ref{PILRefErrorCodes}
for a list of error codes and their explanation.

\paragraph{Parameters}
\begin{itemize}
\item
{\tt char *name [In] } \\
name of the parameter 
\item
{\tt int nelem [In]} \\
number of floats to return
\item
{\tt float *result [Out] } \\
pointer to vector of floats to store result
\end{itemize}


%%%%%%%%%%%       PILGetIntVarVector       %%%%%%%%%%%%%%%%%

\subsubsection{PILGetIntVarVector}

\begin{verbatim}
int PILGetIntVarVector(char *name, int *nelem, int *result); 
\end{verbatim}

\paragraph{Description\\}
This function reads the value of specified parameter. The parameter has
to be of type string (type stored in parameter file). If this is not the
case error code is returned. Ascii string (value of parameter) has to have
exactly *nelem integer numbers separated with spaces. The actual number
of items found in parameter is returned in *nelem.


\paragraph{Return Value\\}
If success returns ISDC\_OK. Error code otherwise. See appendix \ref{PILRefErrorCodes}
for a list of error codes and their explanation.

\paragraph{Parameters}
\begin{itemize}
\item
{\tt char *name [In]} \\
name of the parameter 
\item
{\tt int *nelem [In/Out]} \\
Maximum expected number of item (on input). Actual number of floats found in
parameter (on output)
\item
{\tt int *result [Out]} \\
pointer to vector of integers to store result

\end{itemize}


%%%%%%%%%%%       PILGetRealVarVector       %%%%%%%%%%%%%%%%%

\subsubsection{PILGetRealVarVector}

\begin{verbatim}
int PILGetRealVarVector(char *name, int *nelem, double *result); 
\end{verbatim}

\paragraph{Description\\}
This function reads the value of specified parameter. The parameter has
to be of type string (type stored in parameter file). If this is not the
case error code is returned. Ascii string (value of parameter) has to have
exactly *nelem real numbers separated with spaces.

\paragraph{Return Value\\}
If success returns ISDC\_OK. Error code otherwise. See appendix \ref{PILRefErrorCodes}
for a list of error codes and their explanation. The actual number
of items found in parameter is returned in *nelem.


\paragraph{Parameters}
\begin{itemize}
\item
{\tt char *name [In]} \\
name of the parameter
\item
{\tt int *nelem [In/Out]} \\
Maximum expected number of item (on input). Actual number of floats found in
parameter (on output)
\item
{\tt double *result [Out]} \\
pointer to vector of doubles to store result
\end{itemize}


%%%%%%%%%%%       PILGetReal4VarVector       %%%%%%%%%%%%%%%%%

\subsubsection{PILGetReal4VarVector}

\begin{verbatim}
int PILGetReal4VarVector(char *name, int *nelem, float *result);
\end{verbatim}

\paragraph{Description\\}
This function reads the value of specified parameter. The parameter has
to be of type string (type stored in parameter file). If this is not the
case error code is returned. Ascii string (value of parameter) has to have
at most *nelem real numbers separated with spaces. The actual number
of items found in parameter is returned in *nelem.

\paragraph{Return Value\\}
If success returns ISDC\_OK. Error code otherwise. See appendix \ref{PILRefErrorCodes}
for a list of error codes and their explanation.

\paragraph{Parameters}
\begin{itemize}
\item
{\tt char *name [In] } \\
name of the parameter 
\item
{\tt int *nelem [In/Out]} \\
Maximum expected number of item (on input). Actual number of floats found in
parameter (on output)
\item
{\tt float *result [Out] } \\
pointer to vector of floats to store result
\end{itemize}


%%%%%%%%%%%       PILPutBool       %%%%%%%%%%%%%%%%%

\subsubsection{PILPutBool}

\begin{verbatim}
int PILPutBool(char *name, int b); 
\end{verbatim}

\paragraph{Description\\}
This function sets the value of specified parameter. without any prompts.
The parameter has to be of type
boolean. If this is not the case error code is returned. 

\paragraph{Return Value\\}
If success returns ISDC\_OK. Error code otherwise. See appendix \ref{PILRefErrorCodes}
for a list of error codes and their explanation.

\paragraph{Parameters}
\begin{itemize}
\item
{\tt char *name [In] } \\
name of the parameter 
\item
{\tt int b [In] } \\
new value for argument.
\end{itemize}


%%%%%%%%%%%       PILPutInt       %%%%%%%%%%%%%%%%%

\subsubsection{PILPutInt}

\begin{verbatim}
int PILPutInt(char *name, int i); 
\end{verbatim}

\paragraph{Description\\}
This function sets the value of specified parameter. without any prompts.
The parameter has to be of type integer.
If this is not the case error code is returned. 
The same happens when value passed is out of range.

\paragraph{Return Value\\}
If success returns ISDC\_OK. Error code otherwise. See appendix \ref{PILRefErrorCodes}
for a list of error codes and their explanation.

\paragraph{Parameters}
\begin{itemize}
\item
{\tt char *name [In] } \\
name of the parameter 
\item
{\tt int i [In] } \\
new value for argument.
\end{itemize}


%%%%%%%%%%%       PILPutReal       %%%%%%%%%%%%%%%%%

\subsubsection{PILPutReal}

\begin{verbatim}
int PILPutReal(char *name, double d); 
\end{verbatim}

\paragraph{Description\\}
This function sets the value of specified parameter. without any prompts.
The parameter has to be of type
boolean. If this is not the case error code is returned. 
The same happens when value passed is out of range.

\paragraph{Return Value\\}
If success returns ISDC\_OK. Error code otherwise. See appendix \ref{PILRefErrorCodes}
for a list of error codes and their explanation.

\paragraph{Parameters}
\begin{itemize}
\item
{\tt char *name [In] } \\
name of the parameter 
\item
{\tt double d [In] } \\
new value for argument.
\end{itemize}


%%%%%%%%%%%       PILPutString       %%%%%%%%%%%%%%%%%

\subsubsection{PILPutString}

\begin{verbatim}
int PILPutString(char *name, char *s); 
\end{verbatim}

\paragraph{Description\\}
This function sets the value of specified parameter. without any prompts.
The parameter has to be of type
boolean. If this is not the case error code is returned. 

\paragraph{Return Value\\}
If success returns ISDC\_OK. Error code otherwise. See appendix \ref{PILRefErrorCodes}
for a list of error codes and their explanation.

\paragraph{Parameters}
\begin{itemize}
\item
{\tt char *name [In] } \\
name of the parameter 
\item
{\tt char *s [In] } \\
new value for argument. Before assignment value is truncated PIL\_LINESIZE
characters. 
\end{itemize}


%%%%%%%%%%%       PILPutFname       %%%%%%%%%%%%%%%%%

\subsubsection{PILPutFname}

\begin{verbatim}
int PILPutFname(char *name, char *s); 
\end{verbatim}

\paragraph{Description\\}
This function sets the value of specified parameter. without any prompts.
The parameter has to be of type
boolean. If this is not the case error code is returned. 
The same happens when value passed is out of range.

\paragraph{Return Value\\}
If success returns ISDC\_OK. Error code otherwise. See appendix \ref{PILRefErrorCodes}
for a list of error codes and their explanation.

\paragraph{Parameters}
\begin{itemize}
\item
{\tt char *name [In] } \\
name of the parameter 
\item
{\tt char *s [In] } \\
new value for argument. Before assignment value is truncated PIL\_LINESIZE
characters. 
\end{itemize}


%%%%%%%%%%%       PILGetNumParameters       %%%%%%%%%%%%%%%%%

\subsubsection{PILGetNumParameters}

\begin{verbatim}
int PILGetNumParameters(int *parnum);
\end{verbatim}

\paragraph{Description\\}
This function returns number of parameters in parameter file. The number
returned includes entries for all lines in parameter file, including those
in wrong/invalid format. Actual number of valid parameters can be found
by iteratively calling PILGetParameter and checking for correct format
flag.

\paragraph{Return Value\\}
If success returns ISDC\_OK. Error code otherwise. See appendix \ref{PILRefErrorCodes}
for a list of error codes and their explanation.

\paragraph{Parameters}
\begin{itemize}
\item
{\tt int *parnum [Out] } \\
number of parameters found
\end{itemize}

\paragraph{Notes\\}
{\it
This function does not have its F90 version.
}


%%%%%%%%%%%       PILGetParameter       %%%%%%%%%%%%%%%%%

\subsubsection{PILGetParameter}

\begin{verbatim}
int PILGetParameter(int *parnum); 
\end{verbatim}

\paragraph{Description\\}
This function returns full record about n-th parameter in parameter file.
This record, stored in memory, is maintained internally by PIL. Any changes
to a given parameter, are first reflected in this record and disk update
is done only by PILClose or PILFlushParameters.

\paragraph{Return Value\\}
If success returns ISDC\_OK. Error code otherwise. See appendix \ref{PILRefErrorCodes}
for a list of error codes and their explanation.

\paragraph{Parameters}
\begin{itemize}
\item
{\tt idx [In] } \\
index of parameter to return
\item
{\tt pp [Out] } \\
returned parameter's data
\item
{\tt minmaxok [Out] } \\
flag whether min/max(1) or enum(2) values are defined (and returned) for that parameter
\item
{\tt vmin [Out] } \\
min value for returned parameter (converted to proper type), only when minmaxok==1
\item
{\tt vmax [Out] } \\
max value for returned parameter (converted to proper type), only when minmaxok==1
\end{itemize}

\paragraph{Notes\\}
{\it
One can use the following program to list parameters :

\begin{verbatim}
PILGetNumParameters(&parcnt);
for (i=0; i<parcnt; i++)
 { if (PIL_OK != PILGetParameter(i, &pardata, &minmaxflag, &minval, 
                                 &maxval))
     break;
   if (PIL_FORMAT_OK != pp->format) continue;

   minmaxstr[0] = 0;
   if (1 == minmaxflag) sprintf(minmaxstr, "min=%s, max=%s ", 
                                pp->strmin, pp->strmax);
   if (2 == minmaxflag) sprintf(minmaxstr, "enum=%s ", pp->strmin);

   printf("%-16.16s 0x%02x 0x%02x %20s %s// %s\n", pp->strname, 
          pp->type, pp->mode, pp->strvalue, minmaxstr, pp->strprompt);
 }
\end{verbatim}

Symbolic values are defined in pil.h

This function does not have its F90 version.
}


%%%%%%%%%%%       PILVerifyCmdLine       %%%%%%%%%%%%%%%%%

\subsubsection{PILVerifyCmdLine}

\begin{verbatim}
int PILVerifyCmdLine(void); 
\end{verbatim}

\paragraph{Description\\}
PILVerifyCmdLine scans argument list (given by argc and argv parameters) and
for all parameters in format Name=Value checks whether there is parameter with
such name in parameter table (in memory). If this is not the case it returns
with an error (meaning: bogus parameters specified in command line).

\paragraph{Return Value\\}
If success returns ISDC\_OK. Error code otherwise. See appendix \ref{PILRefErrorCodes}
for a list of error codes and their explanation.

\paragraph{Notes\\}
{\it
This function does not have its F90 version.
}


%%%%%%%%%%%       PILSetRootNameFunction       %%%%%%%%%%%%%%%%%

\subsubsection{PILSetRootNameFunction}\label{PILSetRootNameFunction}

\begin{verbatim}
int PILSetRootNameFunction(int (*func)(char *s));
\end{verbatim}

\paragraph{Description\\}
This function instruct PIL to use function 'func' as a new URL to filename
conversion routine. This function will be called during validation
of any parameter of type file with access checking on ('fr', 'fw', 'fn' or
'fe').

PIL's default function is the one which speaks CFITSIO's language. 

It is valid to call this function with func set to NULL. In this case
any URL will be treated verbatim as a filename during filename 
validation phase.

\paragraph{Return Value\\}
If success returns ISDC\_OK. Error code otherwise. See appendix \ref{PILRefErrorCodes}
for a list of error codes and their explanation.

\paragraph{Parameters}
\begin{itemize}
\item
{\tt func [In] } \\
pointer to the new URL to filename conversion function. This function
should read input URL (passed in s), then it should check if it evaluates 
to filename. If it does it should extract that filename and put it back into s.
New function should return one of the following :
   \begin{itemize}
   \item
   {\tt PIL\_ROOTNAME\_FILE } \\
   if URL evaluates to filename. For instance
   file://some/file.fits[1] or simply /some/file.fits. In both cases
   the filename stored in s will be /some/file.fits (assuming default
   PIL's conversion function).
   \item
   {\tt PIL\_ROOTNAME\_NOTFILE } \\
   if URL does not evaluate to filename
   \item
   {\tt PIL\_ROOTNAME\_STDIN } \\
   if URL specifies standard input stream (stdin/STDIN for instance)
   \item
   {\tt PIL\_ROOTNAME\_STDOUT } \\
   if URL specifies standard input stream (stdout/STDOUT for instance)
   \item
   {\tt PIL\_ROOTNAME\_STDINOUT } \\
   if URL specifies standard input stream ('-' for instance)
   \item
   {\tt other value } \\
   generic error code, like NULL pointers, PIL not initialized, etc...
   \end{itemize}
\end{itemize}

\paragraph{Notes\\}
{\it
This function does not have its F90 version.
}


%%%%%%%%%%%       PILSetFileAccessFunction       %%%%%%%%%%%%%%%%%

\subsubsection{PILSetFileAccessFunction}\label{PILSetFileAccessFunction}

\begin{verbatim}
int PILSetFileAccessFunction(int (*func)(const char *file_name, const char *open_mode)
\end{verbatim}

\paragraph{Description\\}
This function allows the PIL client to supply a custom function for the
purpose of checking file existence and/or access type.

\paragraph{Return Value\\}
If successful returns ISDC\_OK; otherwise returns an error code. See appendix \ref{PILRefErrorCodes}
for a list of error codes and their explanations.

\paragraph{Parameters}
\begin{itemize}
\item
{\tt func [In] } \\
pointer to the new access checking function being supplied by the client. This function
must return a 0 if the file does not exist or does not have the correct access
mode, and a 1 if it does. The supplied function takes two arguments, and must handle
them as follows:
   \begin{itemize}
   \item
   {\tt file\_name} The name of the file being checked.
   \item
   {\tt open\_mode} The access mode to check. This may be either "r", "w", "" or 0.
   The client function should interpret the first two options as checking for
   read and write access, respectively. Either "" or 0 signifies that the client
   function should only check for file existence.
   \end{itemize}
\end{itemize}

\paragraph{Notes\\}
{\it
This function does not have a F90 version.
}


%%%%%%%%%%%       PILSetReadlinePromptMode       %%%%%%%%%%%%%%%%%

\subsubsection{PILSetReadlinePromptMode}\label{PILSetReadlinePromptMode}

\begin{verbatim}
int PILSetReadlinePromptMode(int mode); 
\end{verbatim}

\paragraph{Description\\}
This function changes PIL's promping mode when compiled
with READLINE support. 

\paragraph{Return Value\\}
If success returns ISDC\_OK. Error code otherwise. See appendix \ref{PILRefErrorCodes}
for a list of error codes and their explanation.

\paragraph{Parameters}
\begin{itemize}
\item
{\tt mode [In] } \\
The new PIL prompting mode. Allowable values are:

\begin{itemize}
\item PIL\_RL\_PROMPT\_PIL - standard prompting mode (compatible
with previous PIL versions). The prompt format is :

\begin{verbatim}
           some_text  [ default_value ] : X
\end{verbatim}

: (which is printed) denotes beginning of the edit buffer (which in 
this mode is always initially empty).
X denotes cursor position. To enter empty using this mode, one has to
enter "" (unless redefined by PIL\_EMPTY\_STRING environment variable).

\item PIL\_RL\_PROMPT\_IEB - alternate prompting mode. The prompt
format is :

\begin{verbatim}
           some_text : default_value X
\end{verbatim}

: (which is printed) denotes beginning of the edit buffer which in 
this mode is initially set to the default value.
X denotes cursor position. To enter empty in this mode one
simply has to empty edit buffer using BACKSPACE/DEL keys then 
press RETURN key.
\end{itemize}
\end{itemize}


%%%%%%%%%%%       PILSetLoggerFunction       %%%%%%%%%%%%%%%%%

\subsubsection{PILSetLoggerFunction}\label{PILSetLoggerFunction}

\begin{verbatim}
int PILSetLoggerFunction(int (*func)(char *s)); 
\end{verbatim}

\paragraph{Description\\}
This function instructs PIL to use function 'func' as a new logger
routine. New function will be called whenever PILGetXXX or PILPutXXX
routine fails. The string passed to the new routine (and generated
internally by PIL) usually contains the name of the parameter 
for which the PIL routine failed.

\paragraph{Return Value\\}
If success returns ISDC\_OK. Error code otherwise. See appendix \ref{PILRefErrorCodes}
for a list of error codes and their explanation.

\paragraph{Parameters}
\begin{itemize}
\item
{\tt func [In] } \\
pointer to the new logger function. The 'func' function
should read input text (passed in s), then it should log 
the text somewhere using its own logging mechanism.
Calling PILSetLoggerFunction(NULL) instructs PIL not
to log any messages (this is also done by PILInit).
\end{itemize}

\paragraph{Notes\\}
{\it
The PILSetLoggerFunction function does not have its F90 version.
}


%%%%%%%%%%%       PILSetReprompt       %%%%%%%%%%%%%%%%%

\subsubsection{PILSetReprompt}\label{PILSetReprompt}

\begin{verbatim}
int PILSetReprompt(const char *par_name, int reprompt)
\end{verbatim}

\paragraph{Description\\}
Resets the state of the given parameter such that the next call
to the appropriate PILGet-family function will issue a prompt, regardless of
whether that parameter was supplied on the command line, or was already prompted
for, or is hidden.

\paragraph{Return Value\\}
If successful returns ISDC\_OK; otherwise returns an error code. See appendix \ref{PILRefErrorCodes}
for a list of error codes and their explanation.

\paragraph{Parameters}
\begin{itemize}
\item
{\tt par\_name [In] } \\
Name of the parameter for which to activate reprompt mode.
\item
{\tt reprompt [In] } \\
Whether to reprompt: 1 means reprompt, 0 means do not reprompt.
\end{itemize}

\paragraph{Notes\\}
{ If PILOverrideQueryMode is set to suppress prompts, this function has no effect.
\it
The PILSetReprompt function does not have its F90 version.
}


%%%%%%%%%%%%%%%%%%%%%%%%%%%%%%%%%%%%%%%%%%%%%%%%%%%%%%%%%%

\section{Calling PIL library functions from C/C++}\label{PILRefCcalling}

Applications can use PIL services in one of 2 different modes. The first one
ISDC\_SINGLE\_MODE is the simplest
one. In this mode application simply includes {\tt pil.h} header file, calls
PILInit, plays with parameters by calling
PILGetXxx/PILPutXxx functions and finally shuts down PIL library by calling
PILClose. The skeleton code is given
below. 

\begin{verbatim}

/* this is skeleton code for simple PIL aware applications, this
   is not a working code.
*/

#include <stdio.h>
#include <pil.h>

int main(int argc, char **argv)
{ int r;
  float	fv[5];

r = PILInit(argc, argv);

if (r < 0)
  {
    printf("PILInit failed : %s\n", PIL_err_handler(r));
    return(10);
  }

r = PILGetBool("boolParName1", &intptr);
r = PILGetReal("RealParName34", &doubleptr);
r = PILGetReal4Vector("real4vecname", 5, &(fv[0]));

    /* .... application code follows .... */

PILClose(PIL_OK);
exit(0);
} 

\end{verbatim}

The second mode, called ISDC\_SERVER\_MODE allows for multiple rereads of
parameter file. Using this method
application can exchange data with other processes via parameter file
(provided other processes use locks to
assure exclusive access during read/write operation). One example of code is
as follows : 

\begin{verbatim}

/* this is skeleton code for PIL aware applications running in server mode,
   this is not a working code.
*/

#include <stdio.h>
#include <pil.h>

int main(int argc, char **argv)
{

PILInit(argc, argv);

if (ISDC_SERVER_MODE != PILRunMode)
  exit(-1);  /* error - not in server mode - check parameter file */

for (;;)
 {
   PILReloadParameters();
   PILGetInt("IntParName", &intvar);

   /* place for loop code here
        ...
   */

   PILPutReal("RealParName", 4.567);
   PILFlushParameters();
   if (exit_condition) break;
 }

PILClose(status);
return(PIL_OK);
}

\end{verbatim}

After initial call to PILInit application jumps into main loop. In each
iteration it rereads parameters from file
(there is no need to call PILReloadParameters during first iteration), Based
on new values of just read-in
parameters (which might be modified by another process) application may
decide to exit from loop or continue.
If it decides to continue then after executing application specific loop
code it calls PILFlushParameters to signal
other process that it is done with current iteration. Algorithm described
above is very simple, and the real applications are usually more complicated. 

As mentioned earlier, applications written for ISDC should not use
PILInit/PILClose directly. Instead they should
use CommonInit/CommonExit functions from ISDC's Common Library. 

\paragraph{Notes\\}
{\it
most applications will not support ISDC\_SERVER\_MODE so one can delete
those fragments of skeleton code which deal with this mode.
}


\isdcpart{PIL Fortran 90 Language Interface}
\section{PIL F90 Language API}\label{PILRefF90language}
%
%%%%%%%%%%%%%%%%%%%%%%%%%%%%%%%%%%%%%%%%%%%%%%%%%%%%%%%%%%%%%%%%%%%%%%%%%%%%%%%
%%%%%%%%%%%%%%%%%%%%%%%%%%%%%%%%%%%%%%%%%%%%%%%%%%%%%%%%%%%%%%%%%%%%%%%%%%%%%%%

\subsection{Introduction}\label{PILRefF90intro}

This  section  describes  the  Fortran 90 language implementation of the
Parameter Interface Library Application Programming Interfaces (PIL APIs)
It also gives Fortran 90 languages specific
information necessary to use those APIs. Users interested only
in using PIL from C/C++ applications should see section
\ref{PILRefClanguage}. 

PIL library is standalone and can be compiled independently of other ISDC
libraries.

Unless stated otherwise all PIL API functions return status code of type
int. This is either ISDC\_OK which equals PIL\_OK which equals 0, which
means everything went perfectly or negative value (error code) meaning some
error occured. List of error codes can be found in pil\_f90\_api.f90 file.
Functions given below are the "official" ones. Internally PIL library
calls many more functions. 	

\subsection{PIL Fortran 90 module files}\label{PILRefF90Modules}

Applications calling PIL services from Fortran 90 source code should utilize

\begin{verbatim}
   USE PIL_F90_API
\end{verbatim}

statement to include all PIL definitions. Alternatively they can 

\begin{verbatim}
   USE ISDC_F90_API
\end{verbatim}

which internally includes PIL\_F90\_API module.

PIL\_F90\_API compiled module contains
prototypes of all Fortran 90 API functions, definitions of constants,
declarations of global variables and definitions of data structures.


\subsection{Fortran 90 API functions}\label{PILRefF90functions}

%%%%%%%%%%%          PILinit            %%%%%%%%%%%%%%%%%

\subsubsection{PILINIT}

\begin{verbatim}
FUNCTION PILINIT() 
INTEGER :: PILINIT 
\end{verbatim}

\paragraph{Description\\}
This function initializes PIL library. This function has to be called before
any other PIL function (there are some
exceptions to this rule).  It does the following : \\
First it calculates name of parameter file. Usually name of parameter file
equals GETARG(0)+ ".par" suffix but
this can be overriden by calling PILSETMODULENAME and/or PILSETMODULEVERSION
before calling PILINIT.
After successful termination parameter file is opened and read in, and
global variable PILRunMode is set to one
of the following values : 

\begin{verbatim}
      ISDC_SINGLE_MODE 
      ISDC_SERVER_MODE 
\end{verbatim}

ISDC\_SERVER\_MODE is set whenever there is parameter "ServerMode" and its
value is "yes" or "y". In any other case PILRunMode is set
to ISDC\_SINGLE\_MODE. 

\paragraph{Return Value\\}
If success returns ISDC\_OK. Error code otherwise. See appendix \ref{PILRefErrorCodes}
for a list of error codes and their explanation.

\paragraph{Notes\\}
{\it
Only one parameter file can be open at a time. Parameter file remains open
until PILCLOSE is called.
When writing applications for ISDC one should not use PILINIT directly.
Instead, one should call COMMONINIT
function (from Common library) which calls PILINIT.
}


%%%%%%%%%%%          PILCLOSE            %%%%%%%%%%%%%%%%%

\subsubsection{PILCLOSE}

\begin{verbatim}
FUNCTION PILCLOSE(STATUS) 
INTEGER*4 :: STATUS 
INTEGER :: PILCLOSE 
\end{verbatim}

\paragraph{Description\\}
This function has to be called before application terminates. It closes open
files, and writes all learned
parameters to the disk files (only when status == ISDC\_OK). Once this
function is called one cannot call any
other PIL functions. One can however call PILINIT to reinitialize PIL
library. \\
Function also clears PILRunMode, PILModuleName and PILModuleVersion global
variables. Those global variables
are directly accessible from F90 code. 

\paragraph{Return Value\\}
If success returns ISDC\_OK. Error code otherwise. See appendix \ref{PILRefErrorCodes}
for a list of error codes and their explanation.

\paragraph{Parameters}
\begin{itemize}
\item
{\tt INTEGER*4 :: STATUS [In] } \\
PIL\_OK/ISDC\_OK/0 means normal return, any other value means abnormal
termination. In this case changes to parameters made during runtime
are NOT written to parameter files. 
\end{itemize}

\paragraph{Notes\\}
{\it
this function does not terminate process. It simply shuts down PIL library.
When writing applications for
ISDC one should not use PILCLOSE directly. Instead, one should call
COMMONEXIT function (from Common
library) which calls PILCLOSE. 
}

%%%%%%%%%%%          PILRELOADPARAMETERS            %%%%%%%%%%%%%%%%%

\subsubsection{PILRELOADPARAMETERS}

\begin{verbatim}
FUNCTION PILRELOADPARAMETERS() 
INTEGER :: PILRELOADPARAMETERS 
\end{verbatim}

\paragraph{Description\\}
This function reloads parameters from parameter file. It is called
internally by PILINIT. It should be called
explicitly by applications running in ISDC\_SERVER\_MODE to rescan parameter
file and reload parameters from
it. Current parameter list in memory (including any modifications) is
deleted. PIL library locks whole file for
exclusive access when reading from parameter file. 

\paragraph{Return Value\\}
If success returns ISDC\_OK. Error code otherwise. See appendix \ref{PILRefErrorCodes}
for a list of error codes and their explanation.

\paragraph{Notes\\}
{\it
parameter file remains open until PILCLOSE is called. Function internally
DOES NOT close/reopen parameter file. 
}

%%%%%%%%%%%          PILFLUSHPARAMETERS            %%%%%%%%%%%%%%%%%

\subsubsection{PILFLUSHPARAMETERS}

\begin{verbatim}
FUNCTION PILFLUSHPARAMETERS() 
INTEGER :: PILFLUSHPARAMETERS 
\end{verbatim}

\paragraph{Description\\}
This function flushes changes made to parameter list (in memory) to disk.
Current contents of parameter file is
overwritten. PIL library locks whole file for exclusive access when writing
to parameter file. 

\paragraph{Return Value\\}
If success returns ISDC\_OK. Error code otherwise. See appendix \ref{PILRefErrorCodes}
for a list of error codes and their explanation.

\paragraph{Notes\\}
{\it
parameter file remains open until PILCLOSE is called. 
}

%%%%%%%%%%%          PILSETMODULENAME            %%%%%%%%%%%%%%%%%

\subsubsection{PILSETMODULENAME}

\begin{verbatim}
FUNCTION PILSETMODULENAME(NAME) 
CHARACTER*(*) :: NAME 
INTEGER :: PILSETMODULENAME 
\end{verbatim}

\paragraph{Description\\}
Sets name of the module which uses PIL services. Result is stored in global
variable PILModuleName. Usually
name of parameter file equals argv[0] + ".par" suffix but this can be
overriden by calling PILSETMODULENAME
and/or PILSETMODULEVERSION before calling PILINIT. 

\paragraph{Return Value\\}
If success returns ISDC\_OK. Error code otherwise. See appendix \ref{PILRefErrorCodes}
for a list of error codes and their explanation.

\paragraph{Parameters}
\begin{itemize}
\item
{\tt CHARACTER*(*) :: NAME [In] } \\
new module name
\end{itemize}


%%%%%%%%%%%          PILSETMODULEVERSION            %%%%%%%%%%%%%%%%%

\subsubsection{PILSETMODULEVERSION}

\begin{verbatim}
FUNCTION PILSETMODULEVERSION(VERSION) 
CHARACTER*(*) :: VERSION 
INTEGER :: PILSETMODULEVERSION 
\end{verbatim}

\paragraph{Description\\}
Sets version of the module which uses PIL services. Result is stored in
global variable PILModuleVersion. If
NULL pointer is passed version is set to "version unspecified"

\paragraph{Return Value\\}
If success returns ISDC\_OK. Error code otherwise. See appendix \ref{PILRefErrorCodes}
for a list of error codes and their explanation.

\paragraph{Parameters}
\begin{itemize}
\item
{\tt CHARACTER*(*) :: VERSION [In] } \\
new module version (string)
\end{itemize}


%%%%%%%%%%%          PILGETPARFILENAME            %%%%%%%%%%%%%%%%%

\subsubsection{PILGETPARFILENAME}

\begin{verbatim}
FUNCTION PILGETPARFILENAME(FNAME) 
CHARACTER*(*) :: FNAME 
INTEGER :: PILGETPARFILENAME 
\end{verbatim}

\paragraph{Description\\}
This function retrieves full path of used parameter file. Absolute path is
returned only when PFILES
environment variable contains absolute paths. If parameter file is taken
from current dir then only filename is
returned. 

\paragraph{Return Value\\}
If success returns ISDC\_OK. Error code otherwise. See appendix \ref{PILRefErrorCodes}
for a list of error codes and their explanation.

\paragraph{Parameters}
\begin{itemize}
\item
{\tt CHARACTER*(*) :: FNAME [Out] } \\
name/path of the parameter file in use 
\end{itemize}

\paragraph{Notes\\}
{\it
FNAME buffer should be at least PIL\_LINESIZE characters long. 
}


%%%%%%%%%%%          PILOVERRIDEQUERYMODE            %%%%%%%%%%%%%%%%%

\subsubsection{PILOVERRIDEQUERYMODE}

\begin{verbatim}
FUNCTION PILOVERRIDEQUERYMODE(NEWMODE) 
INTEGER*4 :: NEWMODE 
INTEGER :: PILOVERRIDEQUERYMODE 
\end{verbatim}

\paragraph{Description\\}
This functions globally overrides query mode. When newmode passed is
PIL\_QUERY\_OVERRIDE, prompting for
new values of parameters is completely disabled. If value is bad or out of
range PILGETXXX return immediately
with error without asking user. No i/o in stdin/stdout is done by PIL in
this mode. When newmode is
PIL\_QUERY\_DEFAULT PIL reverts to default query mode. 

\paragraph{Return Value\\}
If success returns ISDC\_OK. Error code otherwise. See appendix \ref{PILRefErrorCodes}
for a list of error codes and their explanation.

\paragraph{Parameters}
\begin{itemize}
\item
{\tt INTEGER*4 :: NEWMODE [In] } \\
new value for query override mode 
\end{itemize}


%%%%%%%%%%%          PILGETBOOL            %%%%%%%%%%%%%%%%%

\subsubsection{PILGETBOOL}

\begin{verbatim}
FUNCTION PILGETBOOL(NAME, RESULT) 
CHARACTER*(*) :: NAME 
INTEGER*4 :: RESULT 
INTEGER :: PILGETBOOL 
\end{verbatim}

\paragraph{Description\\}
This function reads the value of specified parameter. The parameter has to
be of type boolean. If this is not the
case error code is returned. 

\paragraph{Return Value\\}
If success returns ISDC\_OK. Error code otherwise. See appendix \ref{PILRefErrorCodes}
for a list of error codes and their explanation.

\paragraph{Parameters}
\begin{itemize}
\item
{\tt CHARACTER*(*) :: NAME [In] } \\
name of the parameter 
\item
{\tt INTEGER*4 :: RESULT [Out] } \\
integer variable which will store result. Boolean value of FALSE is returned
as a 0. Any other value means TRUE. 
\end{itemize}


%%%%%%%%%%%          PILGETINT            %%%%%%%%%%%%%%%%%

\subsubsection{PILGETINT}

\begin{verbatim}
FUNCTION PILGETINT(NAME, RESULT) 
CHARACTER*(*) :: NAME 
INTEGER*4 :: RESULT 
INTEGER :: PILGETINT
\end{verbatim}

\paragraph{Description\\}
This function reads the value of specified parameter. The parameter has to
be of type integer. If this is not the
case error code is returned. 

\paragraph{Return Value\\}
If success returns ISDC\_OK. Error code otherwise. See appendix \ref{PILRefErrorCodes}
for a list of error codes and their explanation.

\paragraph{Parameters}
\begin{itemize}
\item
{\tt CHARACTER*(*) :: NAME [In] } \\
name of the parameter 
\item
{\tt INTEGER*4 :: RESULT [Out] } \\
integer variable which will store result.
\end{itemize}


%%%%%%%%%%%          PILGETREAL            %%%%%%%%%%%%%%%%%

\subsubsection{PILGETREAL}

\begin{verbatim}
FUNCTION PILGETREAL(NAME, RESULT) 
CHARACTER*(*) :: NAME 
REAL*8 :: RESULT 
INTEGER :: PILGETREAL
\end{verbatim}

\paragraph{Description\\}
This function reads the value of specified parameter. The parameter has to
be of type real. If this is not the
case error code is returned. 

\paragraph{Return Value\\}
If success returns ISDC\_OK. Error code otherwise. See appendix \ref{PILRefErrorCodes}
for a list of error codes and their explanation.

\paragraph{Parameters}
\begin{itemize}
\item
{\tt CHARACTER*(*) :: NAME [In] } \\
name of the parameter 
\item
{\tt REAL*8 :: RESULT [Out] } \\
REAL*8 variable which will store result.
\end{itemize}


%%%%%%%%%%%          PILGETREAL4            %%%%%%%%%%%%%%%%%

\subsubsection{PILGETREAL4}

\begin{verbatim}
FUNCTION PILGETREAL4(NAME, RESULT) 
CHARACTER*(*) :: NAME 
REAL*4 :: RESULT 
INTEGER :: PILGETREAL4
\end{verbatim}

\paragraph{Description\\}
This function reads the value of specified parameter. The parameter has to
be of type real. If this is not the
case error code is returned. 

\paragraph{Return Value\\}
If success returns ISDC\_OK. Error code otherwise. See appendix \ref{PILRefErrorCodes}
for a list of error codes and their explanation.

\paragraph{Parameters}
\begin{itemize}
\item
{\tt CHARACTER*(*) :: NAME [In] } \\
name of the parameter 
\item
{\tt REAL*4 :: RESULT [Out] } \\
REAL*4 variable which will store result.
\end{itemize}

\paragraph{Notes\\}
{\it
Function merely calls PILGETREAL, then converts REAL*8 to REAL*4
}


%%%%%%%%%%%          PILGETSTRING            %%%%%%%%%%%%%%%%%

\subsubsection{PILGETSTRING}

\begin{verbatim}
FUNCTION PILGETSTRING(NAME, RESULT) 
CHARACTER*(*) :: NAME 
CHARACTER*(*) :: RESULT 
INTEGER :: PILGETSTRING
\end{verbatim}

\paragraph{Description\\}
This function reads the value of specified parameter. The parameter has to
be of type string. If this is not the case error code is returned. 
It is possible to enter empty string (without accepting default value).
By default entering string "" (two doublequotes) sets value of given string
parameter to an empty string. If PIL\_EMPTY\_STRING environment variable
is defined then its value is taken as an empty string equivalent.


\paragraph{Return Value\\}
If success returns ISDC\_OK. Error code otherwise. See appendix \ref{PILRefErrorCodes}
for a list of error codes and their explanation.

\paragraph{Parameters}
\begin{itemize}
\item
{\tt CHARACTER*(*) :: NAME [In] } \\
name of the parameter 
\item
{\tt CHARACTER*(*) :: RESULT [Out] } \\
character array which will store result. The character array should have at
least PIL\_LINESIZE characters to assure enough storage for the longest possible string. 
\end{itemize}


%%%%%%%%%%%          PILGETFNAME            %%%%%%%%%%%%%%%%%

\subsubsection{PILGETFNAME}

\begin{verbatim}
FUNCTION PILGETFNAME(NAME, RESULT) 
CHARACTER*(*) :: NAME 
CHARACTER*(*) :: RESULT 
INTEGER :: PILGETFNAME
\end{verbatim}

\paragraph{Description\\}
This function reads the value of specified parameter. The parameter has to
be of type filename. If this is not the case error code is returned. 
It is possible to enter empty string (without accepting default value).
By default entering string "" (two doublequotes) sets value of given string
parameter to an empty string. If PIL\_EMPTY\_STRING environment variable
is defined then its value is taken as an empty string equivalent.

\paragraph{Return Value\\}
If success returns ISDC\_OK. Error code otherwise. See appendix \ref{PILRefErrorCodes}
for a list of error codes and their explanation.

\paragraph{Parameters}
\begin{itemize}
\item
{\tt CHARACTER*(*) :: NAME [In] } \\
name of the parameter 
\item
{\tt CHARACTER*(*) :: RESULT [Out] } \\
character array which will store result. The character array should have at
least PIL\_LINESIZE characters to assure enough storage for the longest possible string. 
\end{itemize}

\paragraph{Notes\\}
{\it
leading and trailing spaces are trimmed.
If type of parameter specifies it, access mode checks are done on file. So
if access mode specifies write
mode, and file is read-only then PILGETFNAME will not accept this filename
and will prompt user to enter name of writable file. 
Before applying any checks the value of the parameter (which may be URL, 
for instance http://, file://etc/passwd) is converted to the filename
(if it is possible). Details are given in paragraph \ref{PILSetRootNameFunction}.
}


%%%%%%%%%%%          PILGETDOL            %%%%%%%%%%%%%%%%%

\subsubsection{PILGETDOL}

\begin{verbatim}
FUNCTION PILGETDOL(NAME, RESULT) 
CHARACTER*(*) :: NAME 
CHARACTER*(*) :: RESULT 
INTEGER :: PILGETDOL
\end{verbatim}

\paragraph{Description\\}
This function reads the value of specified parameter. The parameter has to
be of type string. If this is not the case error code is returned. 
It is possible to enter empty string (without accepting default value).
By default entering string "" (two doublequotes) sets value of given string
parameter to an empty string. If PIL\_EMPTY\_STRING environment variable
is defined then its value is taken as an empty string equivalent.

\paragraph{Return Value\\}
If success returns ISDC\_OK. Error code otherwise. See appendix \ref{PILRefErrorCodes}
for a list of error codes and their explanation.

\paragraph{Parameters}
\begin{itemize}
\item
{\tt CHARACTER*(*) :: NAME [In] } \\
name of the parameter 
\item
{\tt CHARACTER*(*) :: RESULT [Out] } \\
character array which will store result. The character array should have at
least PIL\_LINESIZE characters to assure enough storage for the longest possible string. 
\end{itemize}

\paragraph{Notes\\}
{\it
leading and trailing spaces are trimmed
}


%%%%%%%%%%%          PILGETINTVECTOR            %%%%%%%%%%%%%%%%%

\subsubsection{PILGETINTVECTOR}

\begin{verbatim}
FUNCTION PILGETINTVECTOR(NAME, NELEM, RESULT) 
CHARACTER*(*) :: NAME 
INTEGER   :: NELEM
INTEGER*4 :: RESULT 
INTEGER   :: PILGETINTVECTOR
\end{verbatim}

\paragraph{Description\\}
This function reads the value of specified parameter. The parameter has
to be of type string (type stored in parameter file). If this is not the
case error code is returned. Ascii string (value of parameter) has to have
exactly NELEM integer numbers separated with spaces. If there are more
or less then NELEM then error code is returned.

\paragraph{Return Value\\}
If success returns ISDC\_OK. Error code otherwise. See appendix \ref{PILRefErrorCodes}
for a list of error codes and their explanation.

\paragraph{Parameters}
\begin{itemize}
\item
{\tt CHARACTER*(*) :: NAME [In] } \\
name of the parameter 
\item
{\tt INTEGER   :: NELEM [In] } \\
number of integers to return
\item
{\tt INTEGER*4 :: RESULT [Out] } \\
pointer to vector of integers to store result. Actually, one should
pass 1st element of vector say VECNAME(1), and not VECNAME
\end{itemize}


%%%%%%%%%%%          PILGETREALVECTOR            %%%%%%%%%%%%%%%%%

\subsubsection{PILGETREALVECTOR}

\begin{verbatim}
FUNCTION PILGETREALVECTOR(NAME, NELEM, RESULT) 
CHARACTER*(*) :: NAME 
INTEGER   :: NELEM
REAL*8    :: RESULT 
INTEGER   :: PILGETREALVECTOR
\end{verbatim}

\paragraph{Description\\}
This function reads the value of specified parameter. The parameter has
to be of type string (type stored in parameter file). If this is not the
case error code is returned. Ascii string (value of parameter) has to have
exactly NELEM REAL*8 numbers separated with spaces. If there are more
or less then NELEM then error code is returned.

\paragraph{Return Value\\}
If success returns ISDC\_OK. Error code otherwise. See appendix \ref{PILRefErrorCodes}
for a list of error codes and their explanation.

\paragraph{Parameters}
\begin{itemize}
\item
{\tt CHARACTER*(*) :: NAME [In] } \\
name of the parameter 
\item
{\tt INTEGER   :: NELEM [In] } \\
number of integers to return
\item
{\tt REAL*8 :: RESULT [Out] } \\
pointer to vector of REAL*8's to store result. Actually, one should
pass 1st element of vector say VECNAME(1), and not VECNAME
\end{itemize}


%%%%%%%%%%%          PILGETREAL4VECTOR            %%%%%%%%%%%%%%%%%

\subsubsection{PILGETREAL4VECTOR}

\begin{verbatim}
FUNCTION PILGETREAL4VECTOR(NAME, NELEM, RESULT) 
CHARACTER*(*) :: NAME 
INTEGER   :: NELEM
REAL*4    :: RESULT 
INTEGER   :: PILGETREAL4VECTOR
\end{verbatim}

\paragraph{Description\\}
This function reads the value of specified parameter. The parameter has
to be of type string (type stored in parameter file). If this is not the
case error code is returned. Ascii string (value of parameter) has to have
exactly NELEM REAL*4 numbers separated with spaces. If there are more
or less then NELEM then error code is returned.

\paragraph{Return Value\\}
If success returns ISDC\_OK. Error code otherwise. See appendix \ref{PILRefErrorCodes}
for a list of error codes and their explanation.

\paragraph{Parameters}
\begin{itemize}
\item
{\tt CHARACTER*(*) :: NAME [In] } \\
name of the parameter 
\item
{\tt INTEGER   :: NELEM [In] } \\
number of integers to return
\item
{\tt REAL*4 :: RESULT [Out] } \\
pointer to vector of REAL*4's to store result. Actually, one should
pass 1st element of vector say VECNAME(1), and not VECNAME
\end{itemize}

\paragraph{Notes\\}
{\it
Function merely calls PILGETREAL, then converts REAL*8 to REAL*4
}


%%%%%%%%%%%          PILGETINTVARVECTOR            %%%%%%%%%%%%%%%%%

\subsubsection{PILGETINTVARVECTOR}

\begin{verbatim}
FUNCTION PILGETINTVARVECTOR(NAME, NELEM, RESULT) 
CHARACTER*(*) :: NAME 
INTEGER   :: NELEM
INTEGER*4 :: RESULT 
INTEGER   :: PILGETINTVARVECTOR
\end{verbatim}

\paragraph{Description\\}
This function reads the value of specified parameter. The parameter has
to be of type string (type stored in parameter file). If this is not the
case error code is returned. Ascii string (value of parameter) has to have
exactly *nelem integer numbers separated with spaces. The actual number
of items found in parameter is returned in *nelem.

\paragraph{Return Value\\}
If success returns ISDC\_OK. Error code otherwise. See appendix \ref{PILRefErrorCodes}
for a list of error codes and their explanation.

\paragraph{Parameters}
\begin{itemize}
\item
{\tt CHARACTER*(*) :: NAME [In] } \\
name of the parameter 
\item
{\tt INTEGER   :: NELEM [In/Out] } \\
Maximum expected number of item (on input). Actual number of floats found in
parameter (on output)
{\tt INTEGER*4 :: RESULT [Out] } \\
pointer to vector of integers to store result. Actually, one should
pass 1st element of vector say VECNAME(1), and not VECNAME
\end{itemize}


%%%%%%%%%%%          PILGETREALVARVECTOR            %%%%%%%%%%%%%%%%%

\subsubsection{PILGETREALVARVECTOR}

\begin{verbatim}
FUNCTION PILGETREALVARVECTOR(NAME, NELEM, RESULT) 
CHARACTER*(*) :: NAME 
INTEGER   :: NELEM
REAL*8    :: RESULT 
INTEGER   :: PILGETREALVARVECTOR
\end{verbatim}

\paragraph{Description\\}
This function reads the value of specified parameter. The parameter has
to be of type string (type stored in parameter file). If this is not the
case error code is returned. Ascii string (value of parameter) has to have
exactly *nelem integer numbers separated with spaces. The actual number
of items found in parameter is returned in *nelem.

\paragraph{Return Value\\}
If success returns ISDC\_OK. Error code otherwise. See appendix \ref{PILRefErrorCodes}
for a list of error codes and their explanation.

\paragraph{Parameters}
\begin{itemize}
\item
{\tt CHARACTER*(*) :: NAME [In] } \\
name of the parameter 
\item
{\tt INTEGER   :: NELEM [In/Out] } \\
Maximum expected number of item (on input). Actual number of floats found in
parameter (on output)
{\tt REAL*8 :: RESULT [Out] } \\
pointer to vector of REAL*8's to store result. Actually, one should
pass 1st element of vector say VECNAME(1), and not VECNAME
\end{itemize}


%%%%%%%%%%%          PILGETREAL4VARVECTOR            %%%%%%%%%%%%%%%%%

\subsubsection{PILGETREAL4VARVECTOR}

\begin{verbatim}
FUNCTION PILGETREAL4VARVECTOR(NAME, NELEM, RESULT) 
CHARACTER*(*) :: NAME 
INTEGER   :: NELEM
REAL*4    :: RESULT 
INTEGER   :: PILGETREAL4VARVECTOR
\end{verbatim}

\paragraph{Description\\}
This function reads the value of specified parameter. The parameter has
to be of type string (type stored in parameter file). If this is not the
case error code is returned. Ascii string (value of parameter) has to have
exactly *nelem integer numbers separated with spaces. The actual number
of items found in parameter is returned in *nelem.

\paragraph{Return Value\\}
If success returns ISDC\_OK. Error code otherwise. See appendix \ref{PILRefErrorCodes}
for a list of error codes and their explanation.

\paragraph{Parameters}
\begin{itemize}
\item
{\tt CHARACTER*(*) :: NAME [In] } \\
name of the parameter 
\item
{\tt INTEGER   :: NELEM [In/Out] } \\
Maximum expected number of item (on input). Actual number of floats found in
parameter (on output)
{\tt REAL*4 :: RESULT [Out] } \\
pointer to vector of REAL*4's to store result. Actually, one should
pass 1st element of vector say VECNAME(1), and not VECNAME
\end{itemize}

\paragraph{Notes\\}
{\it
Function merely calls PILGETREAL, then converts REAL*8 to REAL*4
}


%%%%%%%%%%%          PILPUTBOOL            %%%%%%%%%%%%%%%%%

\subsubsection{PILPUTBOOL}

\begin{verbatim}
FUNCTION PILPUTBOOL(NAME, VALUE) 
CHARACTER*(*) :: NAME 
INTEGER*4 :: VALUE 
INTEGER :: PILPUTBOOL 
\end{verbatim}

\paragraph{Description\\}
This function sets the value of specified parameter. without any prompts.
The parameter has to be of type
boolean. If this is not the case error code is returned. 
  
\paragraph{Return Value\\}
If success returns ISDC\_OK. Error code otherwise. See appendix \ref{PILRefErrorCodes}
for a list of error codes and their explanation.

\paragraph{Parameters}
\begin{itemize}
\item
{\tt CHARACTER*(*) :: NAME [In] } \\
name of the parameter 
\item
{\tt INTEGER*4 :: VALUE [In] } \\
new value for argument. 0 means FALSE, any other value means TRUE 
\end{itemize}

%%%%%%%%%%%          PILPUTINT            %%%%%%%%%%%%%%%%%

\subsubsection{PILPUTINT}

\begin{verbatim}
FUNCTION PILPUTINT(NAME, VALUE) 
CHARACTER*(*) :: NAME 
INTEGER*4 :: VALUE 
INTEGER :: PILPUTINT 
\end{verbatim}

\paragraph{Description\\}
This function sets the value of specified parameter. without any prompts.
The parameter has to be of type
integer. If this is not the case error code is returned. The same happens
when value passed is out of range. 

\paragraph{Return Value\\}
If success returns ISDC\_OK. Error code otherwise. See appendix \ref{PILRefErrorCodes}
for a list of error codes and their explanation.

\paragraph{Parameters}
\begin{itemize}
\item
{\tt CHARACTER*(*) :: NAME [In] } \\
name of the parameter 
\item
{\tt INTEGER*4 :: VALUE [In] } \\
new value for argument. 
\end{itemize}


%%%%%%%%%%%          PILPUTREAL            %%%%%%%%%%%%%%%%%

\subsubsection{PILPUTREAL}

\begin{verbatim}
FUNCTION PILPUTREAL(NAME, VALUE) 
CHARACTER*(*) :: NAME 
REAL*8 :: VALUE 
INTEGER :: PILPUTREAL 
\end{verbatim}

\paragraph{Description\\}
This function sets the value of specified parameter. without any prompts.
The parameter has to be of type real.
If this is not the case error code is returned. The same happens when value
passed is out of range

\paragraph{Return Value\\}
If success returns ISDC\_OK. Error code otherwise. See appendix \ref{PILRefErrorCodes}
for a list of error codes and their explanation.

\paragraph{Parameters}
\begin{itemize}
\item
{\tt CHARACTER*(*) :: NAME [In] } \\
name of the parameter 
\item
{\tt REAL*8 :: VALUE [In] } \\
new value for argument. 
\end{itemize}


%%%%%%%%%%%          PILPUTSTRING            %%%%%%%%%%%%%%%%%

\subsubsection{PILPUTSTRING}

\begin{verbatim}
FUNCTION PILPUTSTRING(NAME, VALUE) 
CHARACTER*(*) :: NAME 
CHARACTER*(*) :: VALUE 
INTEGER :: PILPUTSTRING 
\end{verbatim}

\paragraph{Description\\}
This function sets the value of specified parameter. without any prompts.
The parameter has to be of type
string. If this is not the case error code is returned. 
  
\paragraph{Return Value\\}
If success returns ISDC\_OK. Error code otherwise. See appendix \ref{PILRefErrorCodes}
for a list of error codes and their explanation.

\paragraph{Parameters}
\begin{itemize}
\item
{\tt CHARACTER*(*) :: NAME [In] } \\
name of the parameter 
\item
{\tt CHARACTER*(*) :: VALUE [In] } \\
new value for argument. Before assignment value is truncated PIL\_LINESIZE
characters.
\end{itemize}


%%%%%%%%%%%          PILPUTFNAME            %%%%%%%%%%%%%%%%%

\subsubsection{PILPUTFNAME}

\begin{verbatim}
FUNCTION PILPUTFNAME(NAME, VALUE) 
CHARACTER*(*) :: NAME 
CHARACTER*(*) :: VALUE 
INTEGER :: PILPUTFNAME 
\end{verbatim}

\paragraph{Description\\}
This function sets the value of specified parameter. without any prompts.
The parameter has to be of type
filename. If this is not the case error code is returned. The same happens
when value passed is out of range. 

\paragraph{Return Value\\}
If success returns ISDC\_OK. Error code otherwise. See appendix \ref{PILRefErrorCodes}
for a list of error codes and their explanation.

\paragraph{Parameters}
\begin{itemize}
\item
{\tt CHARACTER*(*) :: NAME [In] } \\
name of the parameter 
\item
{\tt CHARACTER*(*) :: VALUE [In] } \\
new value for argument. Before assignment value is truncated PIL\_LINESIZE
characters.

\end{itemize}


%%%%%%%%%%%       PILSETREADLINEPROMPTMODE       %%%%%%%%%%%%%%%%%

\subsubsection{PILSETREADLINEPROMPTMODE}\label{PILSETREADLINEPROMPTMODE}

\begin{verbatim}
FUNCTION PILSETREADLINEPROMPTMODE(MODE)
INTEGER :: MODE
\end{verbatim}

\paragraph{Description\\}
This function changes PIL's promping mode when compiled
with READLINE support. 

\paragraph{Return Value\\}
If success returns ISDC\_OK. Error code otherwise. See appendix \ref{PILRefErrorCodes}
for a list of error codes and their explanation.

\paragraph{Parameters}
\begin{itemize}
\item
{\tt mode [In] } \\
The new PIL prompting mode. Allowable values are:

\begin{itemize}
\item PIL\_RL\_PROMPT\_PIL - standard prompting mode (compatible
with previous PIL versions). The prompt format is :

\begin{verbatim}
           some_text  [ default_value ] : X
\end{verbatim}

: (which is printed) denotes beginning of the edit buffer (which in 
this mode is always initially empty).
X denotes cursor position. To enter empty using this mode, one has to
enter "" (unless redefined by PIL\_EMPTY\_STRING environment variable).

\item PIL\_RL\_PROMPT\_IEB - alternate prompting mode. The prompt
format is :

\begin{verbatim}
           some_text : default_value X
\end{verbatim}

: (which is printed) denotes beginning of the edit buffer which in 
this mode is initially set to the default value.
X denotes cursor position. To enter empty in this mode one
simply has to empty edit buffer using BACKSPACE/DEL keys then 
press RETURN key.
\end{itemize}
\end{itemize}


\section{Calling PIL library functions from Fortran 90}\label{PILRefF90calling}

Applications can use PIL services in one of 2 different modes. The first one
ISDC\_SINGLE\_MODE is the simplest
one. In this mode application simply includes PIL definitions from compiled
module file ({\tt USE PIL\_F90\_API } statement),
calls PILINIT, plays with parameters by calling
PILGETXXX/PILPUTXXX functions and finally shuts
down PIL library by calling PILCLOSE. The skeleton code is given below. 

\begin{verbatim}

PROGRAM SKELETON

USE PIL_F90_API

integer :: r
REAL*8  :: real8vec(5)

r = PILINIT()
if (ISDC_OK /= r) STOP

r = PILGETBOOL("boolname1", INTVAR)
r = PILGETREAL("realname3", REALVAR)
r = PILGETREALVECTOR("real8vecname", 5, real8vec(1))

! now execute application code ...
r = PILEXIT(ISDC_OK)

END

\end{verbatim}

The second mode, called ISDC\_SERVER\_MODE allows for multiple rereads of
parameter file. Using this method
application can exchange data with other processes via parameter file
(provided other processes use locks to
assure exclusive access during read/write operation). One example of code is
as follows : 
  
  
\begin{verbatim}

PROGRAM SKELETON2

USE PIL_F90_API

integer :: r

r = PILINIT()
if (ISDC_OK /= r) STOP

DO
  r = PILRELOADPARAMETERS()
  r = PILGETBOOL("boolname1", INTVAR)
  r = PILGETREAL("realname3", REALVAR)
  IF (BREAKCONDITION) BREAK

! now execute loop code ...

  r = PILPUTINT("intname45", INTVAR)
  r = PILFLUSHPARAMETERS()

  IF  (BREAKCONDITION) BREAK

ENDDO

r = PILEXIT(ISDC_OK)
END

\end{verbatim}

After initial call to PILINIT application jumps into main loop. In each
iteration it rereads parameters from file
(there is no need to call PILRELOADPARAMETERS during first iteration), Based
on new values of just read-in
parameters (which might be modified by another process) application may
decide to exit from loop or continue.
If it decides to continue then after executing application specific loop
code it calls PILFLUSHPARAMETERS to
signal other process that it is done with current iteration. Algorithm
described above is very simple, and it real
applications can be much more complicated. 

As mentioned earlier, applications written for ISDC should not use
PILINIT/PILCLOSE directly. Instead they
should use COMMONINIT/COMMONEXIT functions from ISDC's Common Library. 

\paragraph{Notes\\}
{\it
most applications will not support ISDC\_SERVER\_MODE so one can delete
those fragments of skeleton code which deal with this mode. 
}


\isdcpart{Appendices}
\appendix
%%%%%%%%%%%%%%%%%%%%%%%%%%%%%%%%%%%%%%%%%%%%%%%%%%%%%%%%%%%%%%%%%%%%%%%%%%%%%%%
%
\section{PIL Error Codes}\label{PILRefErrorCodes}
%
%%%%%%%%%%%%%%%%%%%%%%%%%%%%%%%%%%%%%%%%%%%%%%%%%%%%%%%%%%%%%%%%%%%%%%%%%%%%%%%

Error codes (symbolic names and values) are common to both C/C++ and Fortran 90
API. Whenever text in the following table mentions C/C++ function, the same
holds for its Fortran 90 equivalent.


\begin{tabular}{|p{6.2cm}|p{1cm}|p{6.5cm}|}
\hline
  {\bf  Error  Symbol} &
  {\bf Code} &
  {\bf Error Description} \\
\hline
  ISDC\_OK &
  0 &
  No  error \\
\hline
  PIL\_OK &
  0 &
  No  error (alias of ISDC\_OK) \\
\hline
  PIL\_NUL\_PTR &
  -3000 &
  Null pointer passed as an argument, and function does not allow it. Can
  appear in many situations. For example passing NULL as a name of argument
  or passing NULL as a pointer to result variable causes PILGetXXX
  function to return this error code.
  \\
\hline
  PIL\_BAD\_ARG &
  -3001 &
  Bad argument passed. This error may be returned in many situations.
  Usually means that data type is incorrect, i.e. calling PILGetInt("a",..)
  when parameter "a" is of type string. Also when check for file access mode
  fails (PILGetFname) this error is returned. When PIL\_QUERY\_OVERRIDE mode
  is in effect and default value of parameter is out of range or invalid
  this error is returned (PILGetXXX)
  This error may also mean internal PIL error.
  \\
\hline
  PIL\_NO\_MEM &
  -3002 &
  Not enough memory. Means that PIL library was not able to allocate
  sufficient memory to handle request. This kind of error is very unlikely
  to happen, since PIL allocates very small amounts of memory. If it happens
  it probably means that memory is overwritten by process or machine
  (operating system) is not in good shape.
  \\
\hline
  PIL\_NO\_FILE &
  -3003 &
  Cannot open/create file. This error may be returned by PILInit when
  parameter file cannot be found/created/copied/updated and by PILgetFname
  when testing for files existence.
  \\
\hline
  PIL\_ERR\_FREAD &
  -3004 &
  Read from file failed. Means physical disk i/o error, lock problem, or 
  network problem when parameter file resides on NFS mounted partition. 
  This error may be returned by PILInit, PILFlushParameters,
  PILReloadParameters, PILClose
  \\
\hline
  PIL\_ERR\_FWRITE &
  -3005 &
  Write to file failed. Means physical disk i/o error, disk full, lock problem,
  or network problem when parameter file resides on NFS mounted partition. 
  This error may be returned by PILInit, PILFlushParameters,
  PILReloadParameters, PILClose  
  \\
\hline
\end{tabular}

\begin{tabular}{|p{6.2cm}|p{1cm}|p{6.5cm}|}
\hline
  {\bf  Error  Symbol} &
  {\bf Code} &
  {\bf Error Description} \\
\hline
  PIL\_EOS &
  -3006 &
  Unexpected end of string. This error means that given line in parameter
  file is badly formatted and does not contain 7 fields separated by ','.
  This error is handled internally by PIL library and should never be
  returned to user, since PIL library ignores all badly formatted lines
  in parameter files.
  \\
\hline
  PIL\_BAD\_NAME &
  -3007 &
  Invalid name of parameter. This error means that name of parameter found
  in parameter file contains invalid characters (ASCII-7 only are allowed).
  \\
\hline
  PIL\_BAD\_TYPE &
  -3008 &
  Invalid type of parameter in parameter file. This error means that PIL was
  not able to decode parameter type. Probably parameter file is badly
  formatted.
  \\
\hline
  PIL\_BAD\_MODE &
  -3009 &
  Invalid mode of parameter in parameter file. This error means that PIL was
  not able to decode parameter mode. Probably parameter file is badly
  formatted.
  \\
\hline
  PIL\_BAD\_LINE &
  -3010 &
  Invalid line in parameter file encountered. This error means that format
  of line in parameter file does not even resemble correct one.
  \\
\hline
  PIL\_NOT\_IMPLEMENTED &
  -3011 &
  Feature not implemented. This error means that type/mode/indirection type
  is valid (according to IRAF/XPI standard) but PIL library does not
  implement this. Example of unimplemented feature is indirection of data from 
  other parameter files.
  \\
\hline
  PIL\_FILE\_NOT\_EXIST &
  -3012 &
  File does not exist. Included for compatibility reasons. Version 1.0 of
  PIL does not return this error.
  \\
\hline
  PIL\_FILE\_EXIST &
  -3013 &
  File exists. Included for compatibility reasons. Version 1.0 of
  PIL does not return this error.
  \\
\hline
  PIL\_FILE\_NO\_RD &
  -3014 &
  File is not readable. Check file access permission/ownership and mount
  options.
  \\
\hline
  PIL\_FILE\_NO\_WR &
  -3015 &
  File is not writable. Check file access permission/ownership and mount
  options. By default PIL requires READWRITE access to the parameter file.
  \\
\hline
  PIL\_LINE\_BLANK &
  -3016 &
  Blank line encountered. Handled internally by PIL. Never returned to user.  
  \\
\hline
  PIL\_LINE\_COMMENT &
  -3017 &
  Comment line encountered. Handled internally by PIL. Never returned to user.  
  \\
\hline
  PIL\_LINE\_ERROR &
  -3018 &
  Invalid line encountered. Handled internally by PIL. Never returned to user.  
  \\
\hline
\end{tabular}

\begin{tabular}{|p{6.2cm}|p{1cm}|p{6.5cm}|}
\hline
  {\bf  Error  Symbol} &
  {\bf Code} &
  {\bf Error Description} \\
\hline
  PIL\_NOT\_FOUND &
  -3019 &
  No such parameter. This error means that no parameter of given name is
  in parameter file.  
  \\
\hline
  PIL\_PFILES\_TOO\_LONG &
  -3020 &
  PFILES environment variable too long. Included for compatibility reasons.
  Version 1.0 of PIL does not return this error.
  \\
\hline
  PIL\_PFILES\_FORMAT &
  -3021 &
  PFILES environment variable is badly formatted. Included for compatibility reasons.
  Version 1.0 of PIL does not return this error.
  \\
\hline
  PIL\_LOCK\_FAILED &
  -3022 &
  Cannot (un)lock parameter file. This error means that PIL library couldn't
  obtain exclusive access to file (by locking it). Operation still was
  performed, by consistency of data od disk is not assured.
  \\
\hline
  PIL\_BOGUS\_CMDLINE &
  -3023 &
  Bogus parameters found in command line. In other words some of parameter
  names specified in command line do not have their counterparts in 
  parameter file. This error may be returned by PILVerifyCmdLine().
  \\
\hline
  PIL\_NO\_LOGGER &
  -3024 &
  Error code used internally by PIL to signal that no logger function
  has been registered.
  \\
\hline
  PIL\_LINE\_TOO\_MANY &
  -3025 &
  Format error in parameter file. Too many (>7) comma separated
  items found.
  \\
\hline
  PIL\_LINE\_TOO\_FEW &
  -3026 &
  Format error in parameter file. Too few (<7) comma separated
  items found.
  \\
\hline
  PIL\_LINE\_UNMATCHED\_QUOTE &
  -3027 &
  Format error in parameter file. Unbalanced quote or doublequote
  character found.
  \\
\hline
  PIL\_LINE\_NO\_LF &
  -3028 &
  Format error in parameter file. No terminating LineFeed character
  found.
  \\
\hline
  PIL\_LINE\_EXTRA\_SPACES &
  -3029 &
  Format error in parameter file. Extra spaces following closing
  quote or doublequote character found.
  found.
  \\
\hline
  PIL\_BAD\_VAL\_BOOL &
  -3030 &
  Cannot convert string to boolean value. Expecting either yes or no
  string.
  \\
\hline
  PIL\_BAD\_VAL\_INT &
  -3031 &
  Cannot convert string to integer value. Expecting [+-]dddddd format
  \\
\hline
  PIL\_BAD\_VAL\_REAL &
  -3032 &
  Cannot convert string to real value. Expecting [+-]ddd[.ddd][e[+-ddd]] format
  \\
\hline
  PIL\_BAD\_VAL\_INT\_VAR\_VECTOR &
  -3033 &
  Cannot convert string to a vector of integers. Expecting space separated
  list of integer numbers.
  \\
\hline
\end{tabular}

\begin{tabular}{|p{6.2cm}|p{1cm}|p{6.5cm}|}
\hline
  PIL\_BAD\_VAL\_INT\_VECTOR &
  -3034 &
  Cannot convert string to a vector of integers. Expecting space separated
  list of integer numbers. This error is also returned when number of
  integers found in the string does not correspond to the number given
  in a call to PILGetIntVector();
  \\
\hline
  PIL\_BAD\_VAL\_REAL\_VAR\_VECTOR &
  -3035 &
  Cannot convert string to a vector of reals. Expecting space separated
  list of real numbers.
  \\
\hline
  PIL\_BAD\_VAL\_REAL\_VECTOR &
  -3036 &
  Cannot convert string to a vector of reals. Expecting space separated
  list of real numbers. This error is also returned when number of
  reals found in the string does not correspond to the number given
  in a call to PILGetRealVector() or PILGetReal4Vector().
  \\
\hline
  PIL\_OFF\_RANGE &
  -3037 &
  Value is off limits set by min and max fields. This error may be returned
  when PIL\_QUERY\_OVERRIDE mode is in effect.
  \\
\hline
  PIL\_BAD\_ENUM\_VALUE &
  -3038 &
  Value does not match any value from enumerated list (min field). This
  error may be returned when PIL\_QUERY\_OVERRIDE mode is in effect.
  \\
\hline
  PIL\_BAD\_FILE\_ACCESS &
  -3039 &
  Effective file access mode does not match the one specified in parameter
  type (i.e. file is not readable when type is "fr"). This error may be
  returned when PIL\_QUERY\_OVERRIDE mode is in effect.

  \\
\hline
\end{tabular}




\end{document}
%%%%%%%%%%%%%%%%%% END OF TEXT %%%%%%%%%%%%%%%%%%%%%%
